\definecolor{fondTI}{HTML}{869286}

\serie{Heures, minutes, secondes}



\begin{exercice}
Convertis en heures et minutes :\\
78 min ; 134 min ; 375 min ; 35 min ; 3\,840 s.
\end{exercice}


\begin{exercice}
Effectue les calculs :
\begin{enumerate} 
 \item 3 h 25 min $+$ 5 h 33 min ;
 \item 12 h 28 min $-$ 9 h 17 min ;
 \item 6 h 38 min $+$ 19 h 53 min ;
 \item 21 h 15 min $-$ 9 h 29 min ;
 \item 5 h 13 min 33 s $+$ 9 h 45 min 47 s ;
 \item 9 h 6 min 15 s $-$ 8 h 39 min 36 s.
 \end{enumerate}
\end{exercice}


\begin{exercice}
Pose et effectue les opérations suivantes :
\begin{enumerate} 
 \item 18 h 15 min 22 s $+$ 9 h 37 min 43 s ;
 \item 12 h 26 min 52 s $-$ 7 h 39 min 57 s ;
 \item 9 h 38 min 22 s $+$ 4 h 59 min 34 s ;
 \item 12 h 40 min 21 s $-$ 6 h 35 s.
 \end{enumerate}
\end{exercice}


\begin{exercice}
Pose et effectue les opérations suivantes :
\begin{enumerate} 
 \item 13 h 25 min 42 s $+$ 12 h 35 min 52 s ;
 \item 15 h 43 min 08 s $-$ 6 h 51 min 34 s ;
 \item 10 h 41 s $+$ 9 h 57 min 49 s ;
 \item 21 h $-$ 17 h 31 min 32 s.
 \end{enumerate}
\end{exercice}


\begin{exercice}
Un randonneur part en promenade à 9 h 30. Il rentre à 12 h 05, ne s'étant arrêté pour se reposer que lors de trois pauses de 5 min chacune. Pendant combien de temps ce randonneur a‑t‑il marché ?
\end{exercice}


\begin{exercice}
Pierre part de chez lui à 9 h 55 pour aller faire des courses. Il met 12 min pour se rendre au supermarché et il y reste pendant 1 h 35 min.
\begin{enumerate} 
 \item À quelle heure repart‑il du supermarché ?
 \item Il rentre ensuite chez lui et y arrive à 12 h 01. Combien de temps son trajet de retour a‑t‑il duré ?
 \end{enumerate}
\end{exercice}


\begin{exercice}
Sarah a noté les heures de lever et de coucher du Soleil en septembre 2008. Le $1^{er}$ septembre, le Soleil s'est levé à 7 h 09 et il s'est couché à 20 h 31. Le 30 septembre, le Soleil s'est levé à 7 h 50 et il s'est couché à 19 h 30. De quelle durée les jours ont‑ils diminué au mois de septembre 2008 ?
\end{exercice}
 
