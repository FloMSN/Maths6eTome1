
\usepackage{esvect,cancel} 
\newcommand{\chapeaumelon}[1]{\stackrel{\Large \frown}{#1}}

%%%%%%%% pour les figures en tikz
\usepackage{tikz,xparse}%xparse ajouté pour le compas
\usepackage{tkz-tab,tkz-euclide}
\usetkzobj{all}
\usepackage{pgf}
\usetikzlibrary{calc} %pour mettre des calculs dans les coordonnées de points
\usetikzlibrary{arrows}
\usetikzlibrary{patterns}  
\usetikzlibrary{intersections}%pour l'utilisation de l'intersection de 2 figures
\usetikzlibrary{shapes.geometric}
\usepackage{tikzpeople}
%\usepackage{tikzducks}

\definecolor{CyanTikz40}{cmyk}{.4,0,0,0}
\definecolor{CyanTikz20}{cmyk}{.2,0,0,0}

\definecolor{B1prime}      {cmyk}{0.00, 1.00, 0.00, 0.50}
\definecolor{H1prime}      {cmyk}{0.50, 0.00, 1.00, 0.00}

\tikzstyle{general}         =[font=\fontsize{7.5}{9}\selectfont,line width=0.3mm, >=stealth, x=1cm, y=1cm,line cap=round, line join=round]
\tikzstyle{quadrillage}     =[line width=0.3mm, color=CyanTikz40]
\tikzstyle{quadrillageNIV2} =[line width=0.3mm, color=CyanTikz20]
\tikzstyle{quadrillage55}   =[line width=0.3mm, color=CyanTikz40, xstep=0.5, ystep=0.5]
\tikzstyle{cote}            =[line width=0.3mm, <->]
\tikzstyle{epais}           =[line width=0.5mm, line cap=butt]
\tikzstyle{tres epais}      =[line width=0.8mm, line cap=butt]
\tikzstyle{axe}             =[line width=0.3mm, ->, color=Noir, line cap=rect]
\newcommand{\quadrillageSeyes}[2]{%
  \draw[line width=0.3mm, color=A1!10, ystep=0.2, xstep=0.8] #1 grid #2;
  \draw[line width=0.3mm, color=A1!30, xstep=0.8, ystep=0.8] #1 grid #2;
}

% ajouter pour manuel Flo
\newcommand*\circled[1]{\tikz[baseline=(char.base)]{
	\node[shape=circle,draw,inner sep=1pt] (char) {#1};}}

\newcommand{\axeX}[4][0]{%
  \draw[axe] (#2,#1)--(#3,#1);
  \foreach \x in {#4} {\draw (\x,#1) node {\small $+$};
    \draw (\x,#1) node[below] {\small $\numprint{\x}$};
  }%
}
\newcommand{\axeY}[4][0]{%
  \draw[axe] (#1,#2)--(#1,#3);
  \foreach \y in {#4} {\draw (#1, \y) node {\small $+$};
    \draw (#1, \y) node[left] {\small $\numprint{\y}$};
  }%
}
\newcommand{\axeOI}[3][0]{%
  \draw[axe] (#2,#1)--(#3,#1);
  \draw (1,#1) node {\small $+$};
  \draw (1,#1) node[below] {\small $I$};
}
\newcommand{\axeOJ}[3][0]{%
  \draw[axe] (#1,#2)--(#1,#3);
  \draw (#1, 1) node {\small $+$};
  \draw (#1, 1) node[left] {\small $J$};
}
\newcommand{\axeXgraduation}[2][0]{%
  \foreach \x in {#2} {\draw (\x,#1) node {\small $+$};}%
}
\newcommand{\axeYgraduation}[2][0]{%
  \foreach \y in {#2} {\draw (#1, \y) node {\small $+$};}%
}
\newcommand{\origine}{%
  \draw (0,0) node[below left] {\small $0$};
}
\newcommand{\origineO}{%
  \draw (0,0) node[below left] {$O$};
}
\newcommand{\point}[4]{%
  \draw (#1,#2) node[#4] {$#3$};
}
\newcommand{\pointGraphique}[4]{%
  \draw (#1,#2) node[#4] {$#3$};
  \draw (#1,#2) node {$+$};
}
\newcommand{\pointFigure}[4]{
  \draw (#1,#2) node[#4] {$#3$};
  \draw (#1,#2) node {$\times$};
}
\newcommand{\pointC}[3]{
  \draw (#1) node[#3] {$#2$};
}
\newcommand{\pointCGraphique}[3]{
  \draw (#1) node[#3] {$#2$};
  \draw (#1) node {$+$};
}
\newcommand{\pointCFigure}[3]{
  \draw (#1) node[#3] {$#2$};
  \draw (#1) node {$\times$};
}

\newcommand{\NodeAngle}[2]{%
    %\pgfextra{
        \pgfmathanglebetweenpoints%
            {\pgfpointanchor{#1}{center}}%
            {\pgfpointanchor{#2}{center}}%
            \global\let\MyAngle\pgfmathresult
    }%}

    % #1 premier point              ---- Distance entre 2 nodes ----
    % #2 second point
    % On récupère le résultat dans \MyDist
\makeatletter
\newcommand{\NodeDist}[2]{%
    \pgfpointdiff{\pgfpointanchor{#1}{center}}
                 {\pgfpointanchor{#2}{center}}
    % no need to use a new dimen
    \pgf@xa=\pgf@x
    \pgf@ya=\pgf@y
    % to convert from pt to cm   
    \pgfmathparse{veclen(\pgf@xa,\pgf@ya)/28.45274}
    \global\let\MyDist\pgfmathresult % we need a global macro   
}
\makeatother

% ######################
%     Dessin crayon    #
% ######################

\newcommand{\Crayon}[1]{%
    \begin{scope}[scale=.7,#1]
    \fill[gray!60] (-.2,4.8) -- (.2,4.8)
                -- (.2,.8) --(.1,.65)
                -- (0,.8) -- (-.1,.66)
                -- (-.2,.8) -- cycle ;
    \draw[color=white] (0,4.8) -- (0,.8 );
    \fill[black] (-.2,4.3) -- (0,4.27)
                -- (.2,4.3) -- (.2,4.8) arc(30:150:0.23cm) ;
    \fill[brown!50] (-.2,.8)
        -- (0,0) node[coordinate,pos=0.75](a){}
        -- (.2,.8) node[coordinate,pos=0.25](b){}
        -- (.1,.65) -- (0,.8) -- (-.1,.66) -- cycle;
    \fill[gray] (a) -- (0,0) -- (b) -- cycle;
    \end{scope}
}



% ######################
%   Dessin du compas   #
% ######################

\NewDocumentCommand{\Compas}{smm}{%

    \IfBooleanTF{#1}{%
    % with *
    % keep distance between extemities
    }{%
    % without *
    % calulation of the distance between extemities
    \NodeDist{#2}{#3}
    }

    \NodeAngle{#2}{#3}

    \def\L{6} % taille des branches du compas

    % calcul de l'angle de l'ouverture
    \pgfmathsetmacro{\AngleCP}{asin(\MyDist/(2*\L))}

    \begin{scope}[shift=(#2)]
    \begin{scope}[%
        join=round,
        rotate=\MyAngle,
        shift=(270-\AngleCP:-\L)
        ]

    % branche pointe sèche
    \draw[rotate=-\AngleCP,fill=gray!80]
        (0,0)--(0,-\L)--(-.2,-\L+.8)--(-.2,0)--cycle ;
    \draw[rotate=-\AngleCP,fill=gray!05]
        (0,-\L+.8)--(0,-\L)--(-.2,-\L+.8)--cycle ;

    % branche crayon
    \draw[rotate=\AngleCP,fill=gray!80]
        (0,0)--(0,-\L)--(.2,-\L+.8)--(.2,0)--cycle ;

    \begin{scope}[rotate=\AngleCP,shift={(0,-\L)}]
    \Crayon{rotate=-12}
    \draw[fill=gray!25] (\L/30,\L/5) circle (\L/36) ;
    \fill[gray!5] (\L/30,\L/5)
            -- ++(30:\L/36) arc (30:45:\L/36) -- cycle ;
    \fill[gray!5] (\L/30,\L/5)
            -- ++(210:\L/36) arc (210:225:\L/36) ;
    \filldraw (\L/30,\L/5) circle (.02) ;
    \end{scope}

    % haut du compas
    \draw[fill=gray!80] (-.1,0) rectangle (.1,.7) ;
    \draw[fill=gray!25] (0,0) circle (.25) ;
    \fill[gray!5] (0,0) -- (30:.25) arc (30:45:.25) -- cycle ;
    \fill[gray!5,rotate=180] (0,0) -- (30:.25) arc (30:45:.25) -- cycle ;
    \filldraw (0,0) circle (.05) ;
    \end{scope}
    \end{scope}
}


