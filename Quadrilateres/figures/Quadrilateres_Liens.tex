\begin{tikzpicture}[every node/.style={scale=0.7}]
% définition des styles
\def\couleur{yellow!50}
\tikzstyle{quadri}=[draw,fill=\couleur,text=blue]
\tikzstyle{estun}=[->,>=latex,very thick,dotted]

%%%%% les nœuds %%%%%
%%%Quadrilatère
\node (Q) at (0,2) {Quadrilatère};
\coordinate[shift={(0mm,1mm)}] (Q1) at (Q.north west);
\coordinate[shift={(2mm,-1mm)}] (Q2) at (Q.north east);
\coordinate[shift={(-3mm,1mm)}] (Q3) at (Q.south east);
\coordinate[shift={(2mm,-2mm)}] (Q4) at (Q.south west);
\draw[fill=\couleur] (Q1)--(Q2)--(Q3)--(Q4)--cycle;
\node[blue] (Q) at (0,2) {Quadrilatère};
%%%Parallélogramme
\node[rectangle] (P) at (0,1) {Parallélogramme};
\coordinate[shift={(-3mm,0mm)}] (P1) at (P.north west);
\coordinate[shift={(-3mm,0mm)}] (P2) at (P.north east);
\coordinate[shift={(3mm,0mm)}] (P3) at (P.south east);
\coordinate[shift={(3mm,0mm)}] (P4) at (P.south west);
\draw[fill=\couleur] (P1)--(P2)--(P3)--(P4)--cycle;
\node[color=blue] (P) at (0,1) {Parallélogramme};
%%%Rectangle
\node[rectangle,quadri] (R) at (-1,0) {Rectangle};
%%%Losange
\node[shape=diamond,shape aspect=2,quadri] (L) at (1.5,0) {Losange};
%%%Carré
\node[quadri,minimum size=1.2cm] (C) at (0,-1) {Carré};
%%%Trapèze
\node (T) at (2,2) {Trapèze};
\coordinate[shift={(0mm,0mm)}] (T1) at (T.north west);
\coordinate[shift={(0mm,0mm)}] (T2) at (T.north east);
\coordinate[shift={(4mm,0mm)}] (T3) at (T.south east);
\coordinate[shift={(-2mm,0mm)}] (T4) at (T.south west);
\draw[fill=\couleur] (T1)--(T2)--(T3)--(T4)--cycle;
\node[blue] (T) at (2,2) {Trapèze};

%%%%% les flèches %%%%%
\draw[estun] (P)--(Q);
\draw[estun] (R.north)--(P);
\draw[estun] (L.north west)--(P);
\draw[estun] (C)--(R.south);
\draw[estun] (C)--(L.south west);
\coordinate[shift={(-1mm,0mm)}] (TFl) at (T.west);
\draw[estun] (TFl)--(Q);

%%%%%% la légende %%%%%
\draw[estun] (1,-1.2)--(2.3,-1.2)node[midway,above]{est un};
\end{tikzpicture}