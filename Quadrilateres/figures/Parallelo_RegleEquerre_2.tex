\begin{tikzpicture}[every node/.style={scale=0.7}]
\draw[dashed](-0.8,0)--(1.2,0);
\draw[dashed](-0.8,1.2)--(1,1.2);
\draw[thick,H1,densely dashed,->,>=stealth](-.35,0.05)--(-0.35,0.8); 

\draw[thick] (0,0) node{$\times$} -- (1,0) node{$\times$} -- (0.2,1.2)node{$\times$};
\node[below right] at (0,0){A};
\node[below right] at (1,0){B};
\node[above right] at (0.2,1.2){C};

%%%%%%%%%%%%%%%%%%%%%%%%
%%%%%%%%%%%%%%%%%%%%%%%%
%Début règle !! Si la figure est tournée, il faut
%rajouter l'angle de rotation dans le node des graduation
%pour que les nombres soient écrits correctement
%%%%%%%%%%%%%%%%%%%%%%%%
%%%%%%%%%%%%%%%%%%%%%%%% 

    %Graduation max. de la règle
    \def \Taille {6}
    %Définition de l 'angle de rotation de la règle
    \def \Rotation {-90}
    %Définition du décalage de la règle
    \def \DecalX {-1}
    \def \DecalY {1.5}
    %Couleur des élèments de la règle (sauf le remplissage)
    \def \RegleColor {blue!60}

\begin{scope}[shift={(\DecalX,\DecalY)},rotate=\Rotation,scale=.35]
    % contours de la règle
    \draw[color=\RegleColor, fill =blue!5, opacity=0.5,rounded corners=2pt] (-0.2,0.5) rectangle (\Taille+0.2,-0.5);	%Dont couleur de remplissage
    % graduation 1 mm
    \foreach \a in {0,0.1,...,\Taille}{\draw[color=\RegleColor] (\a,0.5)--(\a,0.42);}
    % graduation 5 mm
    \foreach \a in {0,0.5,...,\Taille}{\draw[color=\RegleColor] (\a,0.42)--(\a,0.35);}
    % graduation et repères 10 mm
    \foreach \a in {0,1,...,\Taille}{\draw[color=\RegleColor] (\a,0.35)--(\a,0.25);}
\end{scope}

%%%%%%%%%%%%%%%%%%%%%%%%
%%%%%%%%%%%%%%%%%%%%%%%%
%Fin de la règle
%%%%%%%%%%%%%%%%%%%%%%%%
%%%%%%%%%%%%%%%%%%%%%%%%

%%%%%%%%%%%%%%%%%%%%%%%%
%%%%%%%%%%%%%%%%%%%%%%%%
%Définition des paramètres de l'équerre
%et de son positionnement
%%%%%%%%%%%%%%%%%%%%%%%%
%%%%%%%%%%%%%%%%%%%%%%%%

\def \xorigine {-0.8}; %abscisse de l'origine de l'équerre posée avec un xshift
\def \yorigine {-0.03}; %ordonnée de l'origine de l'équerre posée avec un yshift
\def \rotation {-90}; %angle de rotation de l'équerre
\def \longueur {4}; %longueur de l'équerre
\def \largeur {2}; %largeur de l'équerre
\def \epaisseur {\longueur * 0.1}; %épaisseur de la partie «colorée» de l'équerre

%%%%%%%%%%%%%%%%%%%%%%%%
%%%%%%%%%%%%%%%%%%%%%%%%
%Tracé de l'équerre
%%%%%%%%%%%%%%%%%%%%%%%%
%%%%%%%%%%%%%%%%%%%%%%%%

\begin{scope}[xshift=\xorigine cm,yshift=\yorigine cm,rotate=\rotation,scale=0.3]

%contour extérieur de l'équerre
\coordinate (A) at (0,0) ; %«origine» de l'équerre
\coordinate (B) at (\largeur,0) ;
\coordinate (C) at (0,\longueur) ;
\draw [gray](A)--(B)--(C)--cycle;


%contour intérieur de l'équerre
\coordinate (D) at (\epaisseur,\epaisseur) ;
\coordinate (E) at ($\largeur*(1,0)-{\largeur * \epaisseur / \longueur}*(1,0)-\epaisseur*(1,0)+\epaisseur*(0,1)$);
\coordinate (F) at ($\epaisseur*(1,0)+\longueur*(0,1)-{2*\longueur * \epaisseur / \largeur}*(0,1)$);
\draw [gray](D)--(E)--(F)--cycle;

%partie colorée de l'équerre
\fill [color=blue!50!gray,opacity=.4,even odd rule] (A)--(B)--(C)--cycle (D)--(E)--(F)--cycle;%l'option even odd rule permet de faire le remplissage entre les 2 zones définies

\end{scope}
%%%%%%%%%%%%%%%%%%%%%%%%
%%%%%%%%%%%%%%%%%%%%%%%%
%Fin de l'équerre
%%%%%%%%%%%%%%%%%%%%%%%%
%%%%%%%%%%%%%%%%%%%%%%%%

%%%%%%%%%%%%%%%%%%%%%%%%
%%%%%%%%%%%%%%%%%%%%%%%%
%Définition des paramètres de l'équerre
%et de son positionnement
%%%%%%%%%%%%%%%%%%%%%%%%
%%%%%%%%%%%%%%%%%%%%%%%%

\def \xorigine {-0.8}; %abscisse de l'origine de l'équerre posée avec un xshift
\def \yorigine {1.18}; %ordonnée de l'origine de l'équerre posée avec un yshift
\def \rotation {-90}; %angle de rotation de l'équerre
\def \longueur {4}; %longueur de l'équerre
\def \largeur {2}; %largeur de l'équerre
\def \epaisseur {\longueur * 0.1}; %épaisseur de la partie «colorée» de l'équerre

%%%%%%%%%%%%%%%%%%%%%%%%
%%%%%%%%%%%%%%%%%%%%%%%%
%Tracé de l'équerre
%%%%%%%%%%%%%%%%%%%%%%%%
%%%%%%%%%%%%%%%%%%%%%%%%

\begin{scope}[xshift=\xorigine cm,yshift=\yorigine cm,rotate=\rotation,scale=0.3]

%contour extérieur de l'équerre
\coordinate (A) at (0,0) ; %«origine» de l'équerre
\coordinate (B) at (\largeur,0) ;
\coordinate (C) at (0,\longueur) ;
\draw [gray](A)--(B)--(C)--cycle;


%contour intérieur de l'équerre
\coordinate (D) at (\epaisseur,\epaisseur) ;
\coordinate (E) at ($\largeur*(1,0)-{\largeur * \epaisseur / \longueur}*(1,0)-\epaisseur*(1,0)+\epaisseur*(0,1)$);
\coordinate (F) at ($\epaisseur*(1,0)+\longueur*(0,1)-{2*\longueur * \epaisseur / \largeur}*(0,1)$);
\draw [gray](D)--(E)--(F)--cycle;

%partie colorée de l'équerre
\fill [color=blue!50!gray,opacity=.4,even odd rule] (A)--(B)--(C)--cycle (D)--(E)--(F)--cycle;%l'option even odd rule permet de faire le remplissage entre les 2 zones définies

\end{scope}
%%%%%%%%%%%%%%%%%%%%%%%%
%%%%%%%%%%%%%%%%%%%%%%%%
%Fin de l'équerre
%%%%%%%%%%%%%%%%%%%%%%%%
%%%%%%%%%%%%%%%%%%%%%%%%

%%%%%%%%%%%%%%%%%%%%%%%%
%%%%%%%%%%%%%%%%%%%%%%%%
%Début crayon
%%%%%%%%%%%%%%%%%%%%%%%%
%%%%%%%%%%%%%%%%%%%%%%%%

\begin{scope}[xshift=-0.8cm,yshift=1.2cm,rotate=190,scale=.15] %le crayon, xshift et yshift pour les coordonnées de la pointe, rotate pour l'orientation du crayon
\def \couleur {black}
\coordinate (O) at (0,0);
\fill[\couleur!40] (-0.2,4.8) -- (0.2,4.8) -- (0.2,0.8) --(0.1,0.65) -- (0,0.8) -- (-0.1,0.66) -- (-0.2,0.8) -- cycle; %corps du crayon
\draw[color=white] (0,4.8) -- (0,0.8); %trait intérieur du crayon
\fill[\couleur!90] (-0.2,4.3) -- (0,4.27) -- (0.2,4.3) -- (0.2,4.8) arc(30:150:0.23cm); %partie haute du crayon
\fill[brown!40] (-0.2,0.8) -- (O)node[coordinate,pos=0.75](a){} -- (0.2,0.8)node[coordinate,pos=0.25](b){} -- (0.1,0.65) -- (0,0.8) -- (-0.1,0.66) -- cycle; %pointe du crayon (partie taillée)
\fill[\couleur!90] (a) -- (O) -- (b) -- cycle; %mine du crayon
\end{scope}

%%%%%%%%%%%%%%%%%%%%%%%%
%%%%%%%%%%%%%%%%%%%%%%%%
%Fin crayon
%%%%%%%%%%%%%%%%%%%%%%%%
%%%%%%%%%%%%%%%%%%%%%%%%

\end{tikzpicture}