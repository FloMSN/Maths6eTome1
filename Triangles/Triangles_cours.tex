
%\section{Une section}

% remarque : pour qu'un mot se retrouve dans le lexique : \MotDefinition{asymptote horizontale}{} 

\prof{les exercices rituels de ce chapitre porteront sur les notations.}

\begin{definition}
Un \MotDefinition{triangle isocèle}{} est un triangle qui a deux côtés égaux;

un \MotDefinition{triangle équilatéral}{} est un triangle qui a trois côtés égaux;

un \MotDefinition{triangle rectangle}{} est un triangle qui a deux côtés perpendiculaires.
\end{definition}



%%%%%%%%%%%%%%%%%%%%%%%%%%%%%%%%%%%%%%%%%%%%%%%%%%

\begin{methode*1}[Construire un triangle]
Pour construire un triangle, il faut toujours connaître 3 informations à propos de ce triangle dont une longueur:
\begin{itemize}
    \item 3 longueurs
    \item 2 longueurs et un angle
    \item 1 longueur et 2 angles
\end{itemize}
Parfois, un angle ou des longueurs sont sous-entendus dans la description du triangle (rectangle, isocèle, équilatéral).
\begin{enumerate}
    \item On commence TOUJOURS par faire un dessin à main levée sur lequel on reporte TOUTES les informations données sous forme de \textcolor{red}{codage}. Ce dessin n'a pas besoin d'être en vraies grandeurs; il sert simplement à rendre les choses plus visuelles.
    \item Pour tracer la figure en vraies grandeurs, on commence TOUJOURS par tracer un segment de longueur donnée.
    \item Les étapes suivantes dépendent des informations données. Un fois la figure tracée, on ne doit pas oublier de reporter le \textcolor{red}{codage}. Il est inutile de reporter les valeurs des longueurs.\\
\end{enumerate}
\exercice
Tracer les dessins à main levée des triangles suivants, sans oublier les codages:\\
\begin{enumerate}
    \item Le triangle ABC tel que AB=5cm, BC=4,5cm et l'angle $\widehat{BAC}=63°$.
    \item Le triangle GAZ tel que GA=3,5cm, AZ=5cm et AC=4cm.
    \item Le triangle BUT tel que BU=6cm, l'angle $\widehat{UBT}=80°$ et $\widehat{BUT}=20°$.
    \item Le triangle RIZ, isocèle en Z tel que $\widehat{IRZ}=35°$ et RI=4cm.
    \item Le triangle POT, rectangle en P tel que PO=4cm et $\widehat{PTO}=50°$
\end{enumerate}
\end{methode*1}
%%%%%%%%%%%%%%%%%%%%%%%%%%%%%%%%%%%%%%%%%%%%%%%%%%%%%%%%%%%%%%%%%%%%%%%%%%%%%%%%%
\begin{methode*1}[Avec 3 longueurs]

\begin{exemple*1}
Construis un triangle $KLM$ tel que $KL = 6$ cm ; $LM = 5$ cm et $KM = 4,5$ cm :

On trace une figure à \textbf{main levée} :
\begin{center} \includegraphics[width=2.3cm]{triangleKML} \end{center}

\begin{tabularx}{\textwidth}{X|X|X}
 \input{./Triangles/figures/ConstruireTriangle_1} &  
\begin{tikzpicture}[scale=0.4,every node/.style={scale=1}]
\clip (-1,-1) rectangle (7,4);%pour enlever les cercles «non tracés»
\draw[thick] (0,0) node [above left]{K}--+(90:0.1)--+(-90:0.1)--(0,0)--(6,0) node [above right]{L}--+(90:0.1)--+(-90:0.1);

\path [name path=cercle 1](0,0) circle (4.5cm);
\path [name path=cercle 2](6,0) circle (5cm);

\draw [name intersections={of=cercle 1 and cercle 2,by={I1,I2}}] 
   [dashed,thick] (I1) -- (6,0) node [midway,sloped,above]{5cm};
\begin{scope}
\clip (I1) circle (1);
\draw [thick](6,0) circle (5cm);
\end{scope}

\coordinate (L) at (6,0);
\begin{scope}[scale=0.65]
\Compas {L}{I1} 
\end{scope}

   
\end{tikzpicture} 
 & 
 
\begin{tikzpicture}[scale=0.4,every node/.style={scale=1}]
\clip (-2,-1) rectangle (7,4);%pour enlever les cercles «non tracés»

\draw[very thick] (0,0) node [above left]{K}--+(90:0.08)--+(-90:0.08)--(0,0)--(6,0) node [above right]{L}--+(90:0.08)--+(-90:0.08);

\path [name path=cercle 1](0,0) circle (4.5cm);
\path [name path=cercle 2](6,0) circle (5cm);

\draw [name intersections={of=cercle 1 and cercle 2,by={I1,I2}}];
\draw [very thick](I1) -- (6,0);
\draw [dashed,very thick] (I1) -- (0,0) node [midway,sloped,above]{4,5 cm};
   
\begin{scope}
\clip (I1) circle (1);
\draw [thick](6,0) circle (5cm);
\draw [thick](0,0) circle (4.5cm);
\end{scope}

\coordinate (K) at (0,0);
\begin{scope}[scale=0.65]
\Compas {K}{I1} 
\end{scope}

   
\end{tikzpicture} 
 \\ 
 On trace un segment $[KL]$ de longueur 6 cm. & Le point $M$ est à 5 cm du point $L$ : il appartient au cercle de centre $L$ et de rayon 5 cm. & Le point $M$ est à 4,5 cm du point $K$ : il appartient au cercle de centre $K$ et de rayon 4,5 cm. \\
\end{tabularx} \\

\end{exemple*1}

\exercice 
Construis un triangle $VOL$ tel que :

$VO = 4$ cm ; $OL = 6,3$ cm et $LV = 3,8$ cm.

\vspace{2cm}
%\correction

\exercice 
Construis un triangle \textbf{équilatéral} $EAU$ de 45 mm de côté.

\vspace{2cm}
%\correction

\exercice 
Construis le triangle $UNO$ \textbf{isocèle} en $U$ avec :

$UN = 8$ cm et $NO = 3,6$ cm.
%\correction
 
\end{methode*1}
%%%%%%%%%%%%%%%%%%%%%%%%%%%%%%%%%%%%%%%%%%%%%%%%%%%%%%%%%%%%%%%%%%%%%%%%%%%%%%%%%
\begin{methode*1}[Avec 2 longueurs et 1 angle]

 \begin{exemple*1}
Construis un triangle $BAS$ tel que :

$AB = 10,4$ cm ; $BS = 8$ cm et $\widehat{ABS} = 99^\circ$ : \\[1em]
\begin{tabularx}{\textwidth}{X|X|X}
 \includegraphics[width=3.3cm]{triangleABS} &  \includegraphics[width=3.5cm]{rapporteurBS} & \includegraphics[width=3.6cm]{regleABS} \\ 
 On effectue une figure à main levée en respectant la nature des angles (aigu ou obtus). & On construit un segment $[SB]$ de 8 cm de longueur. On trace un angle mesurant $99^\circ$ de sommet $B$ et de côté $[BS)$. & On place le point $A$ sur le côté de l'angle à 10,4 cm du point $B$. \\
\end{tabularx} \\

\end{exemple*1}

\exercice
Construis un triangle $LET$ tel que :

$\widehat{ETL} = 55^\circ$ ; $ET = 5$ cm et $TL = 4,3$ cm.
\vspace{4cm}
%\correction

\exercice
Construis un triangle $SEL$ tel que :

$SL = 6,4$ cm ; $\widehat{SLE}= 124^\circ$ et $LE = 7,9$ cm.
\vspace{2cm}
%\correction
 
\end{methode*1}
%%%%%%%%%%%%%%%%%%%%%%%%%%%%%%%%%%%%%%%%%%%%%%%%%%%%%%%%%%%%%%%%%%%%%%%%%%%%%%%%%
\begin{methode*1}[Avec 1 longueur et 2 angles]

 \begin{exemple*1}
Construis le triangle $GAZ$ tel que :

$AZ = 11,2$ cm ; $\widehat{GAZ} = 100^\circ$ et $\widehat{AZG} = 31^\circ$. \\[1em]
\begin{tabularx}{\textwidth}{X|X|X}
 \includegraphics[width=3.3cm]{triangleGAZ} &  \includegraphics[width=3.0cm]{rapporteurAZ} & \includegraphics[width=3.0cm]{rapporteurGAZ} \\ 
 On effectue une figure à main levée en respectant la nature des angles (aigu ou obtus). & On trace un segment $[AZ]$ de longueur 11,2 cm. On construit un angle de sommet $A$, de côté $[AZ)$ et mesurant $100^\circ$. & On construit un angle de sommet $Z$, de côté $[ZA)$ et mesurant $31^\circ$. Les côtés des deux angles se coupent au point $G$. \\
\end{tabularx} \\

\end{exemple*1}

\exercice
Construis le triangle $SUD$ tel que :

$UD = 6$ cm ; $\widehat{SUD} = 65^\circ$ ; $\widehat{SDU} = 36^\circ$.
\vspace{4cm}
%\correction

\exercice
Construis le triangle $EST$ tel que :

$ET = 4,6$ cm ; $\widehat{SET} = 93^\circ$ et $\widehat{ETS} = 34^\circ$.
\vspace{2cm}
%\correction
 
\end{methode*1}

\newpage
%%%%%%%%%%%%%%%%%%%%%%%%%%%%%%%%%%%%%%%%%%%%%%%%%%%%%%%%%%%%%%%%%%%%%%%%%%%%%%%%%
\prof{
\begin{activite}[Manipulation: les droites remarquables dans le triangle]
\underline{Le but:} A l'aide de ficelles, construire d'abord un triangle quelconque puis placer les droites remarquables. Nommer le point d'intersection de ces droites.

\underline{Attention:} Prévoir un lieu suffisamment grand pour que les élèves puissent travailler au sol, par groupe.\\

Constituer 4 groupes de travail.\\
\underline{Matériel(par groupe):}
\begin{itemize}
\item 1 grande corde "fermée"
\item 3 petites cordes colorées
\end{itemize}
\underline{Matériel(pour la classe):}
\begin{itemize}
\item lots étiquettes droites remarquables et points d'intersection
\item un appareil photo
\item le matériel de géométrie pour tracer au tableau
\end{itemize}
\underline{déroulement (version 1: découverte):} Prévoir 2 périodes de cours.\\
Donner à chaque groupe une corde "triangle" et un lot de trois cordes  "droites".
\begin{itemize}
\item \underline{médiatrices:} Demander aux élèves de placer une corde de telle sorte qu'elle coupe un côté du triangle en son milieu, perpendiculairement. Faire les 3 médiatrices. Distribuer une étiquette "médiatrices" et "centre du cercle circonscrit" à chaque groupe. Faire une photo.
\item \underline{médianes:} Demander aux élèves de placer une corde qui passe par un sommet du triangle et qui coupe le côté opposé en son milieu. Faire les 3 médianes. Distribuer une étiquette "médianes" et "centre de gravité" à chaque groupe. Faire une photo.
\item \underline{hauteurs:} Demander aux élèves de placer une corde qui passe par un sommet et qui coupe le côté opposé perpendiculairement. Faire les 3 hauteur. Distribuer une étiquette "hauteurs" et "orthocentre" à chaque groupe. Faire une photo.
\item \underline{bissectrices:} Demander aux élèves de placer une corde qui coupe l'angle en deux parties égales. Faire les 3 hauteurs. Distribuer une étiquette "hauteurs" et "centre du cercle inscrit" à chaque groupe. Faire une photo.\\\\
Le travail du professeur est alors de construire un livret contenant une photo d'illustration de chacune des droites, la définition et un schéma récapitulatif, personnalisé pour chaque groupe.
\end{itemize}
Attention: des explications complémentaires intermédiaires sont indispensables. Notamment préciser la nature de ces droite (demi-droite), en faisant bien attention que les cordes matérialisant les bissectrices partent du sommet de l'angle alors que les autre droites dépassent.
\underline{déroulement (version 2: bilan fin de séance):} Prévoir une période de cours.\\
Donner à chaque groupe une corde "triangle" et un lot de trois cordes  "droites".\\
Distribuer à chaque groupe une étiquette "droites".\\
Trois élèves doivent maintenir au sol les trois sommets du triangle. Les autres élèves du groupe, doivent alors placer le plus précisément possible, les trois droites remarquables indiquées sur leur étiquette.\\
Attention: il faut bien expliquer aux élèves "sommets" qu'ils ne doivent pas bouger sinon le triangle change et tout le travail est à recommencer.\\
Un fois que les trois droites sont matérialisées, les élèves appellent le professeur qui vérifie leur travail et les questionne sur le point d'intersection. On place en suite l'étiquette "p. d'intersection". On prend une photo souvenir!\\\\
Le travail du professeur est alors de construire un livret contenant une photo d'illustration de chacune des droites, la définition et un schéma récapitulatif.


\end{activite}
}



\newpage

%%%%%%%%%%%%%%%%%%%%%%%%%%%%%%%%%%%%%%%%%%%%%%%%%%%%%%%%%%%%%%%%%%%%%%%%%%%%%%%%%
%Médiatrice

 \begin{aconnaitre}
Les médiatrices des trois côtés d'un triangle sont \MotDefinition{concourantes}{}.

Leur point d'intersection est le centre du \MotDefinition{cercle circonscrit}{} au triangle. Ce cercle passe par les trois sommets du triangle.
\end{aconnaitre}

 \vspace{2em}
 
 \begin{methode*1}[Construire le cercle circonscrit à un triangle]
 
\begin{remarque}
Il suffit de tracer les médiatrices de deux côtés pour déterminer le centre du cercle circonscrit.
 \end{remarque}
 
 \begin{exemple*1}
Trace le cercle circonscrit au triangle $PAF$ :
 \begin{tabularx}{\textwidth}{X|X|X}
 \includegraphics[width=3.2cm]{triangleFAP} &  \includegraphics[width=3.2cm]{triangleFAOP} & \includegraphics[width=3.2cm]{triangle_cercleFAOP} \\ 
 On construit la médiatrice du segment $[AP]$. & On construit la médiatrice du segment $[FA]$. Soit $O$ le point d'intersection des deux médiatrices. & Le cercle circonscrit est le cercle de centre $O$ et de rayon $OA$ (ou $OF$ ou $OP$). \\
\end{tabularx} \\

\end{exemple*1}

\exercice
Construis le triangle $FEU$ tel que :

$FE = 6$ cm ; $EU = 3,7$ cm et $UF = 3,5$ cm. Trace le cercle circonscrit au triangle $FEU$.
\vspace{4cm}
%\correction

\exercice
Construis le triangle $EAU$ et son cercle circonscrit sachant que : $EA = 6,1$ cm ; $AU = 3$ cm et $UE = 4,9$ cm.

\vspace{2cm}
%\correction

\end{methode*1}

\begin{definition}
Dans un triangle, une \MotDefinition{médiane}{} est une droite qui passe par un sommet du triangle et par le milieu du côté opposé à ce sommet.

Les trois médianes d'un triangle sont concourantes en un point, noté G, et appelé \MotDefinition{centre de gravité}{} du triangle.
\end{definition}

\vspace{2em}

\begin{methode*1}[Construire le centre de gravité d'un triangle]

\begin{remarque}
Puisque les trois médianes sont concourantes, il suffit d'en tracer deux pour déterminer le centre de gravité.
 \end{remarque}

 \begin{exemple*1}
 Trace la centre de gravité du triangle $ABC$ :
 \begin{tabularx}{\textwidth}{X|X|X}
 \includegraphics[width=3.2cm]{CentreGravite1} &  \includegraphics[width=3.2cm]{CentreGravite2} & \includegraphics[width=3.1cm]{CentreGravite3} \\ 
 On trace le milieu de deux des côtés (ici I et J). & On trace les médianes passant par ces deux milieux & Le centre de gravité G est le point d'intertection des médianes. \\
\end{tabularx} \\

\end{exemple*1}

 
\exercice
Construis le triangle $CLE$ tel que 
$CL = 4,5$ cm ; $CE = 5,2$ cm et $\widehat{CLE} = 78^\circ$ puis trace son centre de gravité.
%\correction


\end{methode*1}


%%%%%%%%%%%%%%%%%%%%%%%%%%%%%%%%%%%%%%%%%%%%%%%%%%
%Hauteurs

\newpage

\begin{definition}
Dans un triangle, une \MotDefinition{hauteur}{} est une droite qui passe par un sommet du triangle et qui est perpendiculaire au côté opposé à ce sommet.

Les trois hauteurs d'un triangle sont concourantes en un point, noté H, et appelé \MotDefinition{orthocentre}{} du triangle.
\end{definition}

\vspace{2em}

\begin{methode*1}[Construire les hauteurs d'un triangle]

 \begin{exemple*1}
 Trace la hauteur relative au côté $[BR]$ :
 \begin{tabularx}{\textwidth}{X|X|X}
 \includegraphics[width=3.2cm]{triangleARB_1} &  \includegraphics[width=3.2cm]{triangleARB_2} & \includegraphics[width=3.1cm]{triangleARB_3} \\ 
 On positionne l'équerre perpendiculairement au côté $[BR]$. & On fait glisser l'équerre jusqu'au point $A$. Il faut parfois prolonger le côté $[BR]$. & La hauteur relative au côté $[BR]$ est la droite perpendiculaire au côté $[BR]$ et passant par $A$. \\
\end{tabularx} \\

\end{exemple*1}

\begin{remarque}
On dit aussi «hauteur issue du sommet $A$» pour nommer la hauteur relative au côté $[BR]$.
 \end{remarque}
 
\exercice
Construis le triangle $CAR$ tel que 
$CA = 4,6$ cm ; $AR = 4,3$ cm et $\widehat{CAR} = 102^\circ$ puis trace la hauteur issue de $R$ et celle issue de $C$.
\vspace{2cm}
%\correction
     
\exercice
Construis un triangle $TAX$ tel que 

$TA = 6,3$ cm ; $\widehat{TAX} = 57^\circ$ et $\widehat{ATX} = 63^\circ$ puis trace ses hauteurs.
\vspace{2cm}
%\correction

\exercice
Construis un triangle $BUS$ tel que :

$BU = 6,4$ cm ; $US = 4,8$ cm et $BS = 8$ cm. Trace les trois hauteurs de ce triangle.
%\correction

\end{methode*1}



%%%%%%%%%%%%%%%%%%%%%%%%%%%%%%%%%%%%%%%%%%%%%%%%%%
%Bissectrices

 \newpage
 
 \begin{aconnaitre}
Les trois bissectrices des angles d'un triangle sont concourantes. 

Leur point d'intersection est le \MotDefinition{centre du cercle inscrit}{} dans le triangle. Ce cercle est tangent aux trois côtés du triangle.
 \end{aconnaitre}
 
 \vspace{2em}
 
 \begin{methode*1}[Centre du cercle inscrit dans un triangle]
 
 \begin{remarque}
Il suffit de tracer les bissectrices de deux angles pour déterminer le centre du cercle inscrit.
 \end{remarque}
 
 \begin{exemple*1}
 Construis un triangle $MER$ et son cercle inscrit de centre $O$ :
 \begin{tabularx}{\textwidth}{X|X|X}
 \includegraphics[width=3.2cm]{triangleMRE} &  \includegraphics[width=3.2cm]{triangleMREK} & \includegraphics[width=3.2cm]{triangle_cercleMREK} \\ 
 On trace les bissectrices de deux des trois angles du triangle $MER$. Elles se coupent en $O$, le centre du cercle inscrit. & On trace la perpendiculaire à $(ME)$ passant par le point $O$. Elle coupe $[ME]$ en $K$. On obtient ainsi un rayon $[OK]$ du cercle inscrit dans le triangle $MER$. & On trace le cercle de centre $O$ passant par $K$. \\
 \end{tabularx} \\

\end{exemple*1}
 
 \exercice
Construis un triangle $RAS$ tel que 

$RA = 7$ cm ; $AS = 8$ cm et $RS = 9$ cm puis son cercle inscrit.
%\correction
 
 \end{methode*1}

