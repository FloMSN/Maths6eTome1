
%%%%%%%%%%%%%%%%%%%%%%%%%%%%%%%%%%%%%%%%%%%%%%%%%%%%%%%%%%%
\begin{activite}[Les deux calculatrices]
 \begin{minipage}{0.6\textwidth}
Hervé et Bruno ont tous deux acheté une calculatrice. Hervé a choisi une calculatrice performante avec laquelle il peut écrire les formules. Bruno, lui, a acheté une petite calculatrice solaire. Ils cherchent à calculer $4 + 3 \times 8$.
Tous les deux appuient successivement sur les touches suivantes :  \\[0.5em]
\begin{tabular}{|c|c|c|c|c|c|c|c|c|c|c|}
\cline{1-1} \cline{3-3}\cline{5-5} \cline{7-7}\cline{9-9} \cline{11-11}
4 & & + & & 3 & & $\times$ & & 8 & & = \\ \cline{1-1} \cline{3-3}\cline{5-5} \cline{7-7}\cline{9-9} \cline{11-11}
\end{tabular} \\[0.5em]
Hervé obtient 28 comme résultat et Bruno obtient 56.
 \end{minipage} \hfill%
  \begin{minipage}{0.2\textwidth}
   \includegraphics[width=3.5cm]{calculette}
   \end{minipage}\\
\begin{partie}
Qui a le bon résultat ?
\end{partie}
\begin{partie}
Les deux calculatrices fonctionnent très bien. Comment expliques-tu ces résultats différents ?
\end{partie}
\begin{partie}
Après réflexion, Bruno a trouvé une méthode pour obtenir le bon résultat avec sa calculatrice solaire. Quelle est cette méthode ?
\end{partie}
\end{activite}
%%%%%%%%%%%%%%%%%%%%%%%%%%%%%%%%%%%%%%%%%%%%%%%%%%%%%%%%%%%
\begin{activite}[Attention à la présentation]
 \begin{partie}
Mélanie et Aïssatou ont effectué le même calcul dont voici le détail ci-dessous. L'une d'entre elles s'est trompée. Indique laquelle et explique son erreur :
\vspace{1em}
\begin{center}
 \begin{tabularx}{.6\linewidth}{X|cX}
  \multicolumn{1}{c|}{Mélanie} & \multicolumn{2}{c}{Aïssatou} \\
  $A = \underline{8 \times 4} - 7 \times 3$ && $A = \underline{8 \times 4} - 7 \times 3$ \\
  $A = \underline{32 - 7} \times 3$ && $A = 32 - \underline{7 \times 3}$ \\
  $A = \underline{25 \times 3}$ && $A = \underline{32 - 21}$ \\
  $A = 75$ && $A = 11$ \\
  \end{tabularx}
\end{center}
\end{partie}
\begin{partie}
Mélanie et Aïssatou ont un second calcul à effectuer dont voici le détail ci-dessous. Aïssatou n'a pas réussi à terminer son calcul. Indique son erreur :
\vspace{1em}
\begin{center}
 \begin{tabularx}{.6\linewidth}{X|cX}
  \multicolumn{1}{c|}{Mélanie} & \multicolumn{2}{c}{Aïssatou} \\
  $A = 18 - \underline{(2 + 3)}$ && $A = 18 - \underline{(2 + 3)}$ \\
  $A = \underline{18 - 5}$ && $A = \underline{5 - 18}$ \\
  $A = 13$ && $A = ??$ \\
  \end{tabularx}
\end{center}
\end{partie}
\end{activite}
%%%%%%%%%%%%%%%%%%%%%%%%%%%%%%%%%%%%%%%%%%%%%%%%%%%%%%%%%%%
\begin{activite}[Avec des barres]
\textbf{Notation :}
L’écriture $\dfrac{10}{(2 + 3)}$ correspond à $10 / (2 + 3)$ ou encore à $10 \div (2 + 3)$.
Autrement dit : $\dfrac{10}{(2 + 3)} = 10 \div 5 = 2$. Le trait horizontal s'appelle la \textbf{barre de fraction}.
\begin{partie}
Écris l'expression suivante $\dfrac{10}{(9 + 1)}$ sans la barre de fraction mais en utilisant des parenthèses puis calcule-la.\dotfill
\end{partie}
\begin{partie}
Dany adore les traits de fraction. Il écrit $\dfrac{10}{\left(9 + \dfrac{8}{7+1}\right)}$. Écris le calcul de Dany sans barres de fraction mais en utilisant des parenthèses puis calcule-le. \dotfill
\dotfill
\end{partie}
\begin{partie}
Essaie de construire, sur le même principe, une expression fractionnaire égale à 1 avec trois barres puis avec quatre barres de fraction.
\end{partie}
\end{activite}
%%%%%%%%%%%%%%%%%%%%%%%%%%%%%%%%%%%%%%%%%%%%%%%%%%%%%%%%%%%
