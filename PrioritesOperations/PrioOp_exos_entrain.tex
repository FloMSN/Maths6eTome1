\serie{Opération principale}

\begin{exercice}
Dans chaque expression, surligne le signe de l'opération principale c'est à dire celle que l'on effectue en DERNIER.
 \begin{colitemize}{2}
 \item A = $2 \times 3 +10$
 \item  B = $(3+9) \times 2$
 \item C = $2+3 \times 8$
 \item D = $ 16 \div 4 + 1$
 \item E = $ 12 - 7 + 4$
 \item F = $ 21 \div (10-3)$
 \end{colitemize}
\end{exercice}

\begin{exercice}
Dans chaque expression, surligne le(s) signe(s) de l' (les) opération(s) principale(s).
 \begin{colitemize}{2}
 \item A = $(9 \times 8 + 2) \div 4$
 \item  B = $25 - 6 \times 2 +3 $
 \item C = $ (3 \times 6)\div(5-4+8)$
 \item D = $ (8 \times 10 + 4 -9) \div 5$
 \item E = $ 6 \times (8-2+9) \div 5$
 \item F = $ [(9-8+7)\times4]\div 2$
 \end{colitemize}
\end{exercice}

\begin{exercice}
Dans chaque expression, surligne le(s) signe(s) de l' (les) opération(s) principale(s).\\
A = $4 + 11\times 12\div 4 -9+3$\\
B = $12 \div 6+11-6+13 \times 4$\\
C = $4 \times (6+6)+2 \div (9-7)$\\
D = $(10+2-8) \times 6 \div 4$\\
E = $7+9-2 \times (6 \div 3)$\\
F = $4 \times 6 + 8 \div (9-5)$\\
\end{exercice}

\serie{Priorité des opérations}

\begin{exercice}
Dans les deux tableaux ci-dessous, associe chaque suite d'opérations à son résultat :
\begin{center}
 \begin{tabularx}{\linewidth}{|r|lXr|l|}
  \cline{1-1}\cline{5-5}
  $3 + 2 \times 5$ & $\times$ & & $\times$ & 3 \\  \cline{1-1}\cline{5-5}
  $15 \times 4 \div 3$ & $\times$ & & $\times$ & 6,6 \\ \cline{1-1}\cline{5-5}
  $19 - 4 \times 4$ & $\times$ & & $\times$ & 13 \\ \cline{1-1}\cline{5-5}
  $50 - 7 \times 4 + 9$ & $\times$ & & $\times$ & 31 \\ \cline{1-1}\cline{5-5}
  $17,7 - 11,7 + 0,3 \times 2$ & $\times$ & & $\times$ & 20 \\ \cline{1-1}\cline{5-5}
  \end{tabularx}
\end{center}

\end{exercice}


\begin{exercice}
Effectue les calculs suivants en soulignant à chaque étape le calcul en cours :
 \begin{colitemize}{2}
 \item A = $41 - 12 - 5$
 
 \dotfill
 
 \dotfill

 \item  B = $24,1 - 0,7 + 9,4$
 
 \dotfill

 \dotfill
 
 \item C = $35 \div 7 - 3$
 
 \dotfill

 \dotfill
 
 \item D = $24 \div 2 \div 3$
 
 \dotfill

 \dotfill
 
 \item E = {\small$58 - 14 + 21 \div 3 - 1$}
 
 \dotfill

 \dotfill
 
 \dotfill

 \dotfill

 \item F = $6 \times 8 - 3 + 9 \times 5$
 
 \dotfill

 \dotfill

 \dotfill

 \dotfill
 \end{colitemize}
\end{exercice}


\begin{exercice}
Calcule mentalement et écrit le résultat :
 \begin{colitemize}{2}
 \item A = $(9 + 5) \times 4$

 \dotfill
 
 \item B = $3 \times (31 - 10)$
 
 \dotfill
 
 \item C = $9 + 5 \times 4$

 \dotfill
 
 \item D = $3 \times 31 - 10$   
 
 \dotfill
   
 \item E = $(9 - 2) \times (4 + 1)$
 
 \dotfill
 
 \item F = $17 - (5 + 3) + 5$
 
 \dotfill
 
 \item G = $(9 \times 9 + 5) \div 2$
 
 \dotfill
   
 \item {\footnotesize H = $[6 - (0,25 \times 4 + 2)] \times 9$}
 
 \dotfill.
 \end{colitemize}
\end{exercice}

%%%%%%%%%%%%%%%%%%%Mise en page
\vspace{2cm}
%%%%%%%%%%%%%%%%%%%%%%%%%%%%%%

\begin{exercice}
Effectue les calculs suivants en soulignant le calcul en cours :
 \begin{colitemize}{2}
\item $A = 14 - 5 + 3$

\dotfill

\dotfill

\item $B = 14 - 5 - 3$

\dotfill

\dotfill
	
\item $C = 14 - 5 \times 2$

\dotfill

\dotfill

\item $D = 24 + 1 \times 5$

\dotfill

\dotfill
	
\item $E = 24 \div 2 - 5$

\dotfill

\dotfill
	
\item $F = 24 + 3 \times 11$

\dotfill

\dotfill

 \end{colitemize}
\end{exercice}


\begin{exercice}
Effectue les calculs suivants en soulignant le calcul en cours :
 \begin{colitemize}{2}
\item $A = 3 \times 4 \div 4$

\dotfill

\dotfill

\item $B = 15 + 27 \div 3$

\dotfill
	
\dotfill

\item $C = 45 \div 5 \times 8$

\dotfill
	
\dotfill

\item $D = 20 \div 5 - 4$

\dotfill
	
\dotfill

\item $E = 24 - 3 \times 7$

\dotfill
	
\dotfill

\item $F = 15 - 5 \div 2$

\dotfill

\dotfill

\end{colitemize}
\end{exercice}


\begin{exercice}
Effectue les calculs suivants en soulignant le calcul en cours :
 \begin{colitemize}{2}
\item $A = 8 \times 3 - 5 \times 4$

\dotfill

\dotfill

\item $B = 60 - 14 + 5 \times 3$

\dotfill

\dotfill

\item $C = 36 - 25 \div 5 + 6$

\dotfill

\dotfill

\item $D = 12 + 3 \times 3 \times 2$

\dotfill

\dotfill
 \end{colitemize}
\end{exercice}


%%%%%%%%%%%%%%%%%%%Mise en page
\vspace*{1em}
\newpage
%%%%%%%%%%%%%%%%%%%%%%%%%%%%%%



\begin{exercice}
Effectue les calculs suivants en soulignant le calcul en cours :
 \begin{colitemize}{2}
\item $A = 25 - ( 8 - 3 ) + 1$

\dotfill

\dotfill

\dotfill

\item $B = 25 - ( 8 - 3 + 1)$

\dotfill

\dotfill

\dotfill

\item $C = 25 - 8 - ( 3 + 1 )$

\dotfill

\dotfill

\dotfill

\item $D = ( 25 - 8 ) - 3 + 1$

\dotfill

\dotfill

\dotfill

\item $E = ( 25 - 8 ) - ( 3 \times 2 )$

\dotfill

\dotfill

\dotfill

\item {\small $F = ( 25 \div 5 + 4 ) + 10 \times 2$}

\dotfill

\dotfill

\dotfill
 \end{colitemize}
\end{exercice}


\begin{exercice}
Effectue les calculs suivants en soulignant à chaque étape le calcul en cours :
 \begin{colitemize}{2}
 \item $A = 53 - (12 + 21)$	
 
 \dotfill

 \dotfill

 \dotfill

 \item {\small $B = 2 + (4,7 - 0,3) \times 10$}	

 \dotfill

 \dotfill

 \dotfill
	
 \item $C= 15 + 25 \times 4 - 13$      	

 \dotfill

 \dotfill

 \dotfill
	
 \item {\small $D = 31 - [8 - (0,8 + 2)]$}
	
 \dotfill

 \dotfill

 \dotfill
	
 \item $E = 27 - (9 + 2 \times 0,5)$	

 \dotfill

 \dotfill

 \dotfill
	
 \item {\small $F = (39 + 10) \times (18 - 11)$}
 
 \dotfill

 \dotfill

 \dotfill
\end{colitemize}
\end{exercice}


\begin{exercice}
Effectue les calculs suivants en soulignant à chaque étape le calcul en cours :
\begin{itemize}
 \item $A = 125 - [21 - (9 + 2)]$
 
 \dotfill		

 \dotfill
 
 \dotfill

%%%%%%%%%%%%%%%%%%%%%%Mise en page
\vspace*{5em}
%%%%%%%%%%%%%%%%%%%%%%%%%%%%%%%%%%


 \item $B = [2 \times (4 \times 8 - 11)] \times 2$	

 \dotfill	

 \dotfill

 \dotfill
	
 \item $C = (22 - 3 \times 6) + (7 - 4) \div 3 + 1 + 9 \times 7$  	

 \dotfill	

 \dotfill	

 \dotfill

 \dotfill
 
 \dotfill
	
 \item $D = 3 \times [14,5 - (0,4 \times 5 + 2,5)]$
	
 \dotfill		

 \dotfill

 \dotfill

 \dotfill
	
 \item $E = (34 - 13) \times [9,4 - (8,2 + 1,2)]$

 \dotfill

 \dotfill

 \dotfill

 \dotfill
	
 \item $F = (15 + 8) \times 4 - [(5 \times 3 + 2 + 3) \times (4 - 2)]$
 
 \dotfill	

 \dotfill
 
 \dotfill

 \dotfill
 \end{itemize}
\end{exercice}

\begin{exercice}
Recopie chaque expression en supprimant seulement les parenthèses qui sont inutiles :

$A = 21 - ( 8 \times 4 )$ \dotfill

$B = 21 \times ( 8 - 4 )$ \dotfill

$C = 21 - ( 8 - 4 )$ \dotfill

$D = ( 21 \times 8 ) - 4$ \dotfill

$E = ( 21 + 8 - 1 ) \div 4$  \dotfill

$F = 21 - ( 8 - 4 \times 2 )$ \dotfill
\end{exercice}


\begin{exercice}
Calcule astucieusement :
\begin{enumerate}
 \item $8,4 + 0,76 + 2,6 + 0,24$ \dotfill
 \item $4 \times 0,49 \times 25$ \dotfill
 \item $1 + 2 + 3 + 4 + 5 + 5 + 4 + 3 + 2 + 1$ \dotfill

 \dotfill

 \item $(20 \times 5 + 11) \div (20 \times 5 + 11)$ \dotfill
 
 \dotfill
 
 \item $(14 \times  31 - 21 \times  17) \times  (2 \times  12 - 24)$ \dotfill
 
 \dotfill
 \end{enumerate}
\end{exercice}


%%%%%%%%%%%%%%%%%%%%%%%%Mise en page
\newpage
%%%%%%%%%%%%%%%%%%%%%%%%%%%%%%%%%%%

\begin{exercice}
Calcule chacune des expressions suivantes :

$A = \dfrac{81}{9} \times 5 - 1$ \dotfill

\dotfill

$B = \dfrac{45}{2 \times 3 - 1}$ \dotfill

\dotfill

\dotfill

$C = \dfrac{27}{3 \times 3} - 1$ \dotfill

\dotfill

\dotfill

$D = \dfrac{17 - 5}{3} + 2$ \dotfill 

\dotfill

\dotfill

$E = 7 \times \dfrac{15 \times 4}{16 - 4} + 2$ \dotfill 

\dotfill

\dotfill

$F = \dfrac{37 - 5 \times 2}{3 \times 9}$ \dotfill

\dotfill

\dotfill
\end{exercice}


%%%%%%%%%%%%%%%%%%%%%%%%%%%%%%%%%%%%%%%%%%%%%%%%%%%

\serie{Vocabulaire}

\begin{exercice}
Traduis chaque phrase par une expression :
\begin{enumerate}
 \item Le quotient de dix-huit par la somme de deux et de huit ;
 \item La différence entre seize et le produit de deux par quatre ;
 \item Le quotient de la différence entre dix-sept et six par six ;
 \item Le produit de la somme de huit et de trois par quatre ;
 \item Le quotient de la somme de vingt-cinq et de sept par le produit de quatre par deux.
 \end{enumerate}
\end{exercice}


\begin{exercice}
Traduis chaque expression par une phrase :
\begin{colenumerate}{2}
 \item $6 \times (25 - 6)$ ;
 \item $(5 + 8) \times 8$ ;
 \item $24 - (7 + 9)$ ;
 \item $15 \div (1 + 7)$ ;
 \item $3 \times 9 - 12 \div 4$ ;
 \item $12 + 3 \times (7 - 2)$.
 \end{colenumerate}
\end{exercice} 


\begin{exercice}
Calcule :
\begin{enumerate}
 \item Le produit de $3,75$ par $34,52$ ;
 \item Le produit de $4,5$ par la somme de $6,73$ et de $67,8$ ;
 \item Le produit de la somme de $34,879$ et de $32,8$ par la différence de $78,45$ et de $6,9$.
 \end{enumerate}
\end{exercice} 





%%%%%%%%%%%%%%%%%%%%%%Mise en page
\vspace*{1.2em}
%%%%%%%%%%%%%%%%%%%%%%%%%%%%%%%%%%

%%%%%%%%%%%%%%%%%%%%%%%%%%%%%%%%%%%%%%%%%%%%%%%%%%%

\serie{Problèmes}

\begin{exercice}
La directrice du centre aéré de Tirloulou achète chaque jour des paquets de biscuits pour le goûter. Chaque carton contient 8 paquets de 20 biscuits. Le tableau ci-dessous indique le nombre de cartons achetés pendant 5 jours :

\begin{center}
\begin{tabularx}{\linewidth}{|c|*{6}{>{\centering \arraybackslash}X|}}
\hline \cellcolor{F3} Lundi & \cellcolor{U2} Mardi & \cellcolor{F3} Mercredi & \cellcolor{U2}Jeudi & \cellcolor{F3} Vendredi \\
\hline \cellcolor{F3} 5 & \cellcolor{U2} 3 & \cellcolor{F3} 5 & \cellcolor{U2} 7 & \cellcolor{F3} 6 \\
\hline
\end{tabularx}
\end{center}

\begin{enumerate}
 \item Exprime le nombre de paquets de biscuits achetés durant ces 5 jours à l'aide :
  \begin{itemize}
   \item d'une somme,
   \item d'un produit ;
   \end{itemize}
 \item Effectue ces deux calculs ;
 \item Combien de biscuits ont été achetés durant ces 5 jours.
 \end{enumerate}
 
\end{exercice}


\begin{exercice}[Alouette]
Voici trois mesures d'un air de musique.\\[1em]
\includegraphics[width=8.2cm]{musique}

Le professeur de musique dit que \includegraphics[width=0.2cm]{note_croche} (croche) vaut 0,5 unité de temps, que \includegraphics[width=0.13cm]{note_noire} (noire) vaut 1 unité de temps et que \includegraphics[width=0.22cm]{note_pointee} (noire pointée) vaut 1,5 unité de temps.

\begin{enumerate}
 \item Compte le nombre de notes de chacune des trois sortes et inscris tes résultats dans un tableau ;
 \item Écris un enchaînement d'opérations pour calculer le nombre d'unités de temps utilisées pour écrire cet air puis calcule ce nombre.
 \end{enumerate}

\end{exercice}


\begin{exercice}[Le bon choix]
Pour chaque problème, choisis l'expression correcte (et donc simplifiée) donnant la solution.\\[-1em]
\begin{enumerate}
 \item Paul avait 35 CHF. Il a dépensé 5 CHF puis gagné six francs. Quelle somme a-t-il dorénavant ?
 \begin{itemize}
  \item $A = 35 - 5 + 6$ ;
  \item $B = (35 - 5) + 6$ ;
  \item $C = 35 - (5 + 6)$.
  \end{itemize}
 \item Lucie a acheté trois crayons à 1,50 CHF et 8 feutres à 2,40 CHF en payant avec un billet de cinquante francs. Quelle somme lui a-t-on rendue ?
  \begin{itemize}
  \item $A = 50 - 3 \times 1,5 - 8 \times 2,4$ ;
  \item $B = 50 - 3 \times 1,5 + 8 \times 2,4$ ;
  \item $C = (50 - 3 \times 1,5) - 8 \times 2,4$.
  \end{itemize}
 \item Après avoir utilisé 6,2 m d'une bobine de fil de 15 m, on réalise 5 morceaux de même longueur finissant ainsi la bobine. Quelle est la longueur commune de ces morceaux ?
  \begin{itemize}
   \item $A = 15 - (6,2 \div 5)$ ;
   \item $B = (15 - 6,2) \div 5$ ;
   \item $C = 15 - 6,2 \div 5$.
   \end{itemize}
 \item Dans une salle il y a 20 couples et 14 célibataires. Combien y a t-il de personnes dans cette salle ?
  \begin{itemize}
   \item $A = (20 + 14) \times 2$ ;
   \item $B = 14 + 2 \times 20$ ;
   \item $C = 14 + (2 \times 20)$.
   \end{itemize}
 \end{enumerate}

\end{exercice}


