
\begin{aconnaitre}
Dans une \MotDefinition{expression}{}, on effectue d'abord les calculs entre les parenthèses les plus intérieures puis les multiplications et les divisions de gauche à droite et, enfin, les additions et les soustractions de gauche à droite.
\end{aconnaitre}

\begin{methode*1}[Calculer une expression (1)]

\begin{exemple*1}
Calcule $A$ = 3 $\times$ (4 + 5 $\times$ 7) + 2 $\times$ 5 $-$ 6.

\begin{center}
\begin{tabularx}{1.2\linewidth}{ccccccccX}
$A$=	 	& 3 $\times$	& (4 +	& 5 $\times$ 7)	& + & 2 $\times$  5	& $-$ 6	& $\rightarrow$ & On effectue les calculs dans les \\ \cline{4-4}
&&&&&&&&  parenthèses. On effectue les\\
&&&&&&&& multiplications qui sont prioritaires. \\
$A$= 	& 3 $\times$  	& (4 + 	&  35)  		&+ & 2 $\times$  5 	&$-$ 6  	& $\rightarrow$ & Dans les parenthèses, on \\ \cline{3-4}
 &&&&&&&& effectue ensuite les additions.\\
$A$= 	& 3 $\times$  	&    \multicolumn{2}{c}{39}      		     		& + & 2 $\times$  5 	&$-$ 6  	& $\rightarrow$ & On effectue les multiplications.\\ \cline{2-3}\cline{6-6}
 &&&&&&&& \\
$A$= 	&      		\multicolumn{2}{c}{117} &             			& +  & 10  			& $-$6  	& $\rightarrow$ & On effectue les additions et les \\ \cline{2-7}
 &&&&&&&& soustractions de gauche\\
 &&&&&&&& à droite.\\
$A$= 	&                \multicolumn{6}{c}{121}                             								&  & \\
\end{tabularx}
\end{center}
\end{exemple*1}



 %\begin{center}
 %\begin{tabularx}{\linewidth}{ccccccX}
 % $A$= & 7  + & 2  $\times$ &  & $-$5 &  $\rightarrow$ &  \\ \cline{4-4}
 % & & & & & & parenthèses \\
 % $A$= & 7  + & 2  $\times$ & 12 & $-$5 &  $\rightarrow$ & On effectue les multiplications. \\ \cline{3-4}
 % & & & & & & \\
 % $A$= & 7  + & \multicolumn{2}{c}{24} & $-$5  & $\rightarrow$ & On effectue les additions et \\ \cline{2-4}
 %  & & & & & & les soustractions de gauche à\\
 % & & & & & & droite.\\
 % $A$= &  \multicolumn{2}{c}{31} & & $-$5 &  $\rightarrow$ & On effectue les additions et\\ \cline{2-5}
 % & & & & & & les soustractions de gauche à\\
 % & & & & & & droite.\\
 % $A$= &  \multicolumn{4}{c}{26}  &  & \\
 % \end{tabularx}
 % \end{center}


\exercice 
Dans les expressions suivantes, entoure le signe de l'opération prioritaire :
\begin{colenumerate}{2}
 \item $7 + 25 \times 2 - 9$ ;
 \item $17 - 2 \times 3 + 5$ ;
 \item $28 - (5 + 6 \times 3)$ ;
 \item $7 \times [4  + (1 + 2) \times 5]$.
 \end{colenumerate}
%\correction
 
 \exercice 
Calcule les expressions suivantes en soulignant les calculs en cours :
\begin{colenumerate}{2}
 \item $18 - 3 + 5$ \dotfill

\dotfill
 \item $45 - 3 \times 7$ \dotfill

\dotfill
 \item $(4 + 3 \times 2) \div 2 - 3$ \dotfill

\dotfill
 \item $120 - (4 + 5 \times 7)$ \dotfill

\dotfill
 \end{colenumerate}
%\correction


\end{methode*1}


\begin{aconnaitre}
Lorsqu’une division est indiquée par une barre de fraction, on calcule séparément ce qui est au-dessus de la barre (le numérateur) et ce qui est au-dessous (le dénominateur), puis on effectue la division.
\end{aconnaitre}



\begin{methode*1}[Calculer une expression (2)]

\begin{exemple*1}
Calcul $B = \dfrac{13 + (1+ 3 \times 4 - 8)}{6 \times 2 - 5 + 1}$ : \\[1em]
$B = \dfrac{13 + (1+ 3 \times 4 - 8)}{6 \times 2 - 5 + 1} = \dfrac{13 + 5}{12 - 4} = \dfrac{18}{8} = 18 \div 8 = 2,25$.
\end{exemple*1}


 \exercice 
Calcule les expressions suivantes :
\begin{colenumerate}{2}
\item $A=\dfrac{15 +9}{5 - 2}$\dotfill

\hfill

\dotfill

\item $B=\dfrac{6 \times 4 + 2}{5 \times 2}$\dotfill

\hfill

\dotfill

\item $C=\dfrac{12 - (9 - 5)}{(7- 5) \times 4}$\dotfill

\hfill

\dotfill

\item $D=\dfrac{(6 - 4) \times (7 - 2)}{8 \times 5 \div 4}$ \dotfill

\hfill

\dotfill

\hfill

\dotfill

 \end{colenumerate}
%\correction

\end{methode*1}

\begin{methode*1}[Les bons mots]
\begin{exemple*1}

Donne les définitions des mots : 
\begin{itemize}
\item somme : \dotfill
\item différence :\dotfill
\item produit :\dotfill
\item quotient :\dotfill
\item terme :\dotfill
\item facteur :\dotfill
\end{itemize}
\end{exemple*1}

\exercice

\begin{enumerate}
\item
Dans chaque expression, entoure le symbole de l'opération que l'on effectue en dernier :
\begin{colitemize}{4}
 \item $A = 5 \times (7 + 9)$ ;
 \item $B = 5 \times 7 + 9$ ;
 \item $C = 9 - 5 + 7$ ;
 \item $D = 5 + 7 - 9$.
 \end{colitemize}


\item
Le professeur demande d'écrire une phrase pour traduire chaque expression. Mélissa a repéré que le début de la phrase correspond à l'opération que l'on effectue en dernier.\\
Par exemple, pour l'expression $A$, la phrase commence par : « Le produit de ... . ».\\
Complète la fin de la phrase pour l'expression $A$.


\item
Écris une phrase pour traduire chacune des expressions $B$, $C$ et $D$.

\end{enumerate}

\end{methode*1}

\prof
qedfijvweofhtwrhvwrfuierg8fe7zrwfghvwrt87gz3hrw