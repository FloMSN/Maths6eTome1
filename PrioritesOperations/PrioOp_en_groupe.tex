
\begin{TP}[Codes secrets]

\partie{Dans un sens}

\begin{enumerate}
 \item Recopiez le tableau dans votre cahier :
 
 \begin{center}
 \begin{tabularx}{\linewidth}{|c|X|c|c|c|}
  \hline
  \rowcolor{A3}  &  &  & Somme des & Lettre \\
  % ci-dessous : on peint d'abord les lignes avec rowcolor puis on fait le multirow en -2 (vers le haut) pour éviter que le rowcolor ne le recouvre
  \rowcolor{A3} \multirow{-2}{*}{Calcul \no} & \multirow{-2}{*}{Expression} & \multirow{-2}{*}{Résultat} &  chiffres & associée \\\hline
  \rowcolor{F3} 1) & $(7 - 5) \times (16 - 9)$ & & & \\\hline
  \rowcolor{A2} 2) & $(3 \times 2 \times 30 + 14) : 2$ & & & \\\hline
  \rowcolor{F3} 3) & $(4 \times 2 \times 9) : (17 - 3 \times 5)$ &  & &\\\hline
  \rowcolor{A2} 4) & $(11 \times (98 + 2) + 11) \times 5$ & &  & \\\hline
  \rowcolor{F3} 5) & $(97 + 4) \times 9 \times (6 - 1)$ & & &  \\\hline
  \rowcolor{A2} 6) & $(23 \times 5 - 1) \times (6 + 4) : 4$ & &  & \\\hline
  \rowcolor{F3} 7) & $(40 \times 4 \times 2 + 4) : (6 + 3)$ & &  &\\\hline
  \rowcolor{A2} 8) & $(101 \times 3 - 2) \times 9 \times 3$ & &  &\\\hline
  \end{tabularx}
\end{center}

 \item Calculez chacune des huit expressions qui sont écrites dans ce tableau (en notant le détail des calculs) puis reportez les résultats dans votre tableau ;
 \item Pour chaque résultat, calculez la somme de ses chiffres et reportez-là dans votre tableau ;
 \item Chaque somme obtenue est associée à une lettre de l'alphabet ($A$ pour 1, $B$ pour 2, $C$ pour 3, ...). Écrivez les huit lettres obtenues dans le tableau ;
 \item Reconstituez alors un mot qui vous est familier, en remettant les lettres dans le bon ordre.
\end{enumerate}

\partie{Dans l'autre sens}

\begin{enumerate}
 \item Vous allez désormais faire le travail dans le sens contraire. Pour cela, reproduisez le tableau de la \textbf{1\up{re} partie} et placez-y les lettres du mot "MATHS" dans la dernière colonne ;
 \item Pour chaque lettre, trouvez la valeur qui lui est associée et inscrivez-la dans la colonne « somme des chiffres » de votre tableau ;
 \item Pour chaque lettre, inventez un calcul dont la somme des chiffres du résultat est la valeur de la lettre (au total, il faudra avoir utilisé au moins deux fois des parenthèses et tous les signes opératoires).
\end{enumerate}


\partie{Et pour finir \ldots}

\begin{enumerate}
 \item Choisissez un mot du vocabulaire mathématique contenant huit lettres puis inventez huit expressions qui permettent de retrouver les huit lettres de ce mot ;
 \item Recopiez ce tableau sur une feuille (et ce tableau uniquement) afin qu'un autre groupe puisse décoder le mot caché en effectuant les calculs.
 
 \end{enumerate}

\end{TP}


%%%%%%%%%%%%%%%%%%%%%%%%%%%%%%%%%%%%%%%%%%%%%%%%%%%%%%%%%%%%

\begin{TP}[Notation Polonaise Inverse]

La Notation Polonaise Inverse (NPI), également connue sous le nom de notation post-fixée, permet de noter les formules arithmétiques sans utiliser de parenthèses.

Cette notation est utilisée par certaines calculatrices, ordinateurs ou logiciels. Pour la suite, « Entrée » signifiera qu'on appuie sur la touche entrée d'une calculatrice utilisant cette notation.

\partie{Découverte}

Nathalie a une calculatrice qui utilise la Notation Polonaise Inverse. Pour effectuer le calcul $5 \times (7 + 3)$, elle tape : \\[-1em]
\begin{center} \boxed{\textcolor{C2}{7}} \quad \boxed{\text{\textcolor{C2}{Entrée}}} \quad \boxed{\textcolor{H1}{3}} \quad \boxed{\text{\textcolor{H1}{Entrée}}} \quad \boxed{\textcolor{BleuOuv}{+}} \quad \boxed{\textcolor{J1}{5}} \quad \boxed{\text{\textcolor{J1}{Entrée}}} \quad \boxed{\times} \end{center}

\vspace{1em}

Voici ce qui s'inscrit sur l'écran de sa calculatrice :\\[1em]
\begin{center} \includegraphics[width=8cm]{ecran} \end{center}

\begin{enumerate}
 \item Essayez de trouver ce qu'il faut taper en NPI pour calculer :
 \begin{itemize}
  \item $A = 8 \times (7 - 5)$ ;
  \item $B = (3,7 + 8) \times 9$ ;
  \item $C = 5 + 3 \times 7$.
  \end{itemize}
  
 \item Recherchez à quels calculs correspondent les saisies suivantes puis effectuez-les :
  \begin{itemize} 
  
  \vspace{1em}
  
  \item \boxed{4} \quad \boxed{\text{Entrée}} \quad \boxed{1} \quad \boxed{\text{Entrée}} \quad $\boxed{-}$ \quad \boxed{12} \quad \boxed{\text{Entrée}} \quad $\boxed{\times}$ \\[-0.75em]
  \item \boxed{25} \quad \boxed{\text{Entrée}} \quad \boxed{8} \quad \boxed{\text{Entrée}} \quad $\boxed{1,5}$ \quad \boxed{\text{Entrée}} \quad $\boxed{\times}$ \quad $\boxed{-}$
  \end{itemize}
\end{enumerate}


\partie{Un peu plus loin}  

\begin{enumerate}
 \item Recherchez à quels calculs correspondent les saisies suivantes puis effectuez-les :
   \begin{itemize} 
  
  \vspace{1em}
  
  \item \boxed{7} \quad \boxed{\text{Entrée}} \quad \boxed{4} \quad \boxed{\text{Entrée}} \quad $\boxed{-}$ \quad \boxed{3} \quad \boxed{\text{Entrée}} \quad $\boxed{\times}$ \quad \boxed{2} \quad \boxed{\text{Entrée}} \quad $\boxed{\times}$\\[-0.75em]
  \item \boxed{8} \quad \boxed{\text{Entrée}} \quad \boxed{3} \quad \boxed{\text{Entrée}} \quad $\boxed{+}$ \quad \boxed{9} \quad \boxed{\text{Entrée}} \quad \boxed{4} \quad \boxed{\text{Entrée}} \quad $\boxed{-}$ \quad $\boxed{\times}$
  \end{itemize}
  
  \vspace{1em}
  
 \item Essayez de trouver ce qu'il faut taper en NPI pour calculer : 
 \begin{itemize}
  \item $D = (18 + 3) \times (17 - 5)$ ;
  \item $E = (((5 - 2) \times 3) - 4) \times 8$ ;
  \item $F = (25 - 4) \times 5 + 8 : 4$.
  \end{itemize}
Inventez cinq calculs différents contenant chacun au moins un couple de parenthèses. Sur votre cahier, effectuez ces calculs puis écrivez sur une feuille la saisie en NPI qui correspond à chacun d'eux afin qu'un autre groupe puisse les effectuer.

 \end{enumerate}
 
\end{TP}

