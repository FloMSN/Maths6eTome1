\prof
Dans ce cours, on prendra soin de faire le lien avec les ensembles de nombres. Faire un rappel grâce que schéma donné au début de cet ouvrage.

\section{Les nombres relatifs}

% remarque : pour qu'un mot se retrouve dans le lexique : \MotDefinition{asymptote horizontale}{} 

\vspace{4em}

\begin{center}
    \begin{tikzpicture}[every node/.style={scale=0.6}]



%contour immeuble
\draw[fill=gray!40!yellow] (0,-3) rectangle (4,6);
%cage ascenseur
\draw[fill=gray!20] (1.5,-3)rectangle (2.5,6);
%étages sous-sol
\foreach \y in {-2,-1,1}{\draw (0,\y)--(1.5,\y) (2.5,\y)--(4,\y);}
%rez de chaussée (couleur différente)
\draw[fill=yellow!30!white] (0,0) rectangle (1.5,1) (2.5,0) rectangle (4,1); 
%niveau du sol
\draw[fill=gray!80] (-1,0) rectangle (5,0.1);
%fenêtres étages
\foreach \y in {1,2,...,5}{\foreach \x in {0.3,0.9,2.8,3.4} {\draw[fill=blue!40] (\x,\y+0.2) rectangle (\x+0.3,\y+0.8);}}
%ascenseur
\draw[fill=white] (1.6,2) rectangle (2.4,3);
\draw (2.2,5.7) circle (2mm);
\draw[very thick](2,3)--(2,5.7);
\draw[very thick] (2,5.7) arc (180:-80:2mm);
%boutons ascenseur étages
\foreach \y in {1,2,...,5}{\node[draw,circle,minimum size=1cm,fill=H2,text=black,scale=1] at (5.5,\y+0.5){\y}; }
\node[draw,circle,minimum size=1cm,fill=A2,text=black,scale=1] at (5.5,0.5){0}; 
\node[minimum width=1.5cm,minimum height=1cm,rounded corners=4pt,draw,rectangle,fill=A2,text=black,scale=1] at (7,0.5){Rez de chaussée}; 
\foreach \y in {-1,-2,-3}{\node[minimum width=1cm,minimum height=1cm,rounded corners=4pt,draw,rectangle,fill=B2,text=black,scale=1] at (5.5,\y+0.5){\y}; }

%insertion personnages
\draw (1.8,2.3) node[santa,minimum size=0.6cm]{} ;
\draw (2.2,2.3) node[mexican,minimum size=0.6cm]{} ;
%\draw (4,0) node[duck,minimum size=1.5cm]{};


\end{tikzpicture}
\end{center}

\begin{definition}
Lorsqu'on effectue un déplacement, deux données sont importantes:
\begin{itemize}
    \item le sens du déplacement
    \item la longueur du déplacement
\end{itemize}

En mathématiques, ce déplacement est caractérisé par \textcolor{C2}{\textbf{un nombre relatif}}.\\
\end{definition}

%%%%%%%%%%%%%%%%%%%%%%%%%%%%%%%%%%%%%%%%%%%%%%%%%%%%%%%%%%%%%%%%%
\begin{definition}
On commence toujours à compter à partir de 0.\\
Lorsque que le nombre est \textbf{inférieur à 0} (à gauche ou en dessous de 0), on parle de \textcolor{C2}{\textbf{nombre négatif}} et on utilise un signe $-$ placé devant le chiffre.\\
Lorsque que le nombre est \textbf{supérieur à 0} (à droite ou en dessus de 0), on parle de \textcolor{C2}{\textbf{nombre positif}} et on peut alors utiliser un signe $+$ placé devant le chiffre mais ce n'est pas toujours obligatoire.\\
0 est le seul nombre à la fois positif et négatif.\\
\end{definition}



%%%%%%%%%%%%%%%%%%%%%%%%%%%%%%%%%%%%%%%%%%%%%%%%%%%%%%%%%%%%%%%%%
\begin{definition}
La valeur du déplacement est donnée par le chiffre placé après le signe. C'est ce qu'on appelle \textcolor{C2}{\textbf{ la valeur absolue}}.
Il s'agit de la distance entre 0 est le nombre relatif. Il existe une notation pour parler de la valeur absolue:|nombre|\\
Deux nombres relatifs qui ne diffèrent \textbf{que} par leur signe sont \textcolor{C2}{\textbf{opposés}}.
\end{definition}

\begin{methode*1}[Trouver la valeur absolue d'un nombre relatif]


\begin{exemple*1}
Donne la valeur absolue du nombre $-2$ :

$|-2|$ = 2.
\end{exemple*1}


\exercice
Donne la valeur absolue des nombres suivants : $+5$ ; $-7$ ; $+64,78$ et $-123,4$.
%\correction

\end{methode*1}


\begin{methode*1}[Savoir utiliser le vocabulaire]



\begin{exemple*1}
Quel est le signe du nombre $-3$ ? Quel est son opposé ? \\[1em]
Le signe de $- 3$ est $-$, il est négatif. Son opposé est $+ 3$ que l'on écrit aussi 3.
\end{exemple*1}

\exercice 
Donne le signe des nombres relatifs suivants :

$+1235$ ; $-587$ ; $0$ ; $-1$ ;  $3,5$ ; $-0,001$.
%\correction

\exercice 
Donne l'opposé des nombres relatifs suivants :

$-2,531$ ; $0$ ; $1,245$ ;  $-0,03$ et $0,003$.
%\correction

\end{methode*1}



%%%%%%%%%%%%%%%%%%%%%%%%%%%%%%%%%%%%%%%%%%%%%%%%%%%%%%%%%%%%%%%%%

\newpage


\begin{aconnaitre}
Tout point d'une droite graduée est repéré par un nombre relatif appelé son \textcolor{C2}{\textbf{abscisse}}.

\begin{tikzpicture}
\draw[->] (-6,0) -- (6,0);
\draw (0,.2) node[above] {$O$} ;
\draw (0,-.2) node[below] {$0$} ;
\draw (1,-.2) node[below] {$+1$} ;
\draw (-4,1) node {$A$} ;
\draw[->] (-4,.8) -- (-4,.2) ;
\draw (0,0) node {$|$} ; \draw (1,0) node {$|$} ; \draw (2,0) node {$|$} ; \draw (3,0) node {$|$} ; \draw (4,0) node {$|$} ; \draw (5,0) node {$|$} ;
\draw (-1,0) node {$|$} ; \draw (-2,0) node {$|$} ; \draw (-3,0) node {$|$} ; \draw (-4,0) node {$|$} ; \draw (-5,0) node {$|$} ;
\end{tikzpicture}

\end{aconnaitre}

\vspace{2em}


\begin{methode*1}[Repérer un point sur une droite graduée]



\begin{exemple*1}
Sur la droite graduée ci-dessus, lis l'abscisse du point $A$ : \\[1em]
\begin{minipage}[c]{0.4\linewidth}
Le point $A$ est à gauche de l'origine :

son abscisse est donc négative.

La distance du point $A$ au point $O$ est $4$.
 \end{minipage} \hfill%
 \begin{minipage}[c]{0.1\linewidth}
 \begin{center}\includegraphics[width=0.23cm]{accolade_droite}\end{center}
  \end{minipage} \hfill%
  \begin{minipage}[c]{0.4\linewidth}
  donc l'abscisse du point $A$ est $-4$.
   \end{minipage} \\
\end{exemple*1}


\begin{exemple*1}
Trace une droite graduée et place les points $B(+6)$ et $C(-5)$ : \\[1em]
\begin{minipage}[c]{0.4\linewidth}
L'abscisse du point $B$ est $+6$ donc
 \end{minipage} \hfill%
 \begin{minipage}[c]{0.1\linewidth}
 \begin{center}\includegraphics[width=0.23cm]{accolade_gauche}\end{center}
  \end{minipage} \hfill%
  \begin{minipage}[c]{0.4\linewidth}
  Son abscisse est positive : le point $B$ est donc à droite de l'origine.
  
  Sa distance à l'origine est de 6 unités.
  \end{minipage} \\[0.5em]

\begin{minipage}[c]{0.4\linewidth}
L'abscisse du point $C$ est $-5$ donc
 \end{minipage} \hfill%
 \begin{minipage}[c]{0.1\linewidth}
 \begin{center}\includegraphics[width=0.23cm]{accolade_gauche}\end{center}
  \end{minipage} \hfill%
  \begin{minipage}[c]{0.4\linewidth}
  Son abscisse est négative : le point $C$ est donc à gauche de l'origine. 
  
  Sa distance à l'origine est de 5 unités.
   \end{minipage} \\

\begin{tikzpicture}
\draw[->] (-5.5,0) -- (6.5,0);
\draw (0,.2) node[above] {$O$} ;
\draw (0,-.2) node[below] {$0$} ;
\draw (1,-.2) node[below] {$+1$} ;
\draw (-5,1) node {$C$} ; \draw[->] (-5,.8) -- (-5,.2) ;
\draw (6,1) node {$B$} ; \draw[->] (6,.8) -- (6,.2) ;
\draw (0,0) node {$|$} ; \draw (1,0) node {$|$} ; \draw (2,0) node {$|$} ; \draw (3,0) node {$|$} ; \draw (4,0) node {$|$} ; \draw (5,0) node {$|$} ;\draw (6,0) node {$|$} ;
\draw (-1,0) node {$|$} ; \draw (-2,0) node {$|$} ; \draw (-3,0) node {$|$} ; \draw (-4,0) node {$|$} ; \draw (-5,0) node {$|$} ;
\end{tikzpicture}

\end{exemple*1}


\exercice
Trace une droite graduée d'origine $O$, une unité valant 2 cm. Places-y les points $A$, $B$, $C$, $D$ et  $E$, $F$ d'abscisses respectives $+3$ ; $-2$ ; $+5$ ; $-3$ et $-1,5$ ; $+2,5$. Que peux-tu dire des abscisses de $A$ et $D$ ?
%\correction

\end{methode*1}

%%%%%%%%%%%%%%%%%%%%%%%%%%%%%%%%%%%%%%%%%%%%%%%%%%%%%%%%%%%%%%%%

\section{Comparaison}

\vspace{4em}

\begin{aconnaitre}
\MotDefinition{Comparer deux nombres}{}, c'est trouver lequel est le plus grand (ou le plus petit) ou dire s'ils sont égaux.
\end{aconnaitre}

\vspace{4em}

\begin{methode*1}[Comparer deux nombres relatifs]


\begin{exemple*1}
Compare 9,37 et 92,751 puis 81,36 et 81,357 :

On compare d'abord les \textbf{\textcolor{H1}{parties entières}} des deux nombres :
\begin{itemize}
 \item $\textbf{\textcolor{H1}{9}} < \textbf{\textcolor{H1}{92}}$ donc $9,37 < 92,751$.
 \item $81,357$ et $81,36$ ont la même partie entière. On compare alors les \textbf{\textcolor{B2}{parties décimales}} : $81,357 = 81+0,357$ et $81,36=81+0,36$ mais $0,36=0,360$.
 \end{itemize}
Or \textbf{\textcolor{B2}{360 millièmes}} est plus grand que \textbf{\textcolor{B2}{357 millièmes}} donc $81,36 > 81,357$.
\end{exemple*1}


\begin{exemple*1}
Écris un encadrement de 1,564 au dixième : \\[0.5em]
$1,564 = 1 + 0,500 + 0,064$ et 0,064 est plus petit que 1 dixième. Ainsi, 1,564 est compris entre $1 + 0,5$ et $1 + 0,5 + 0,1$ , soit $1 + 0,6$. \\[0.5em]
Donc un encadrement au dixième de 1,564 est : $1,5 < 1,564 < 1,6$.
\end{exemple*1}

\exercice
Compare les nombres suivants :
\begin{colenumerate}{3}
 \item $+5$ et $+9$ ;
 \item $-3$ et $+8$ ;
 \item $-6$ et $-12$ ;
 \item $-5$ et $-9$ ;
 \item $5,1$ et $-5,3$ ;
 \item $-6,2$ et $-6,4$.
 \end{colenumerate}
%\correction

\end{methode*1}


\newpage

\begin{aconnaitre}
\textbf{Deux nombres relatifs positifs} sont rangés dans l'ordre de leur valeur absolue.

Un \textbf{nombre relatif négatif} est inférieur à un \textbf{nombre relatif positif}.

\textbf{Deux nombres relatifs négatifs} sont rangés dans l'ordre inverse de leur valeur absolue.
\end{aconnaitre}

\vspace{4em}


\begin{methode*1}[Comparer des nombres relatifs]

\begin{exemple*1}
Compare les nombres $-9$ et $-7$ : \\[0.5em]
\begin{tabular}{ccl} 
 $-9$ et $-7$ & $\longrightarrow$ & On veut comparer deux nombres relatifs négatifs. \\
 $9 > 7$ & $\longrightarrow$ & On détermine les valeurs absolues de $-9$ et de $-7$ puis  \\
& & on les compare. \\
 $-9 < -7$ & $\longrightarrow$ & On range les nombres $-9$ et $-7$ dans l'ordre inverse de leur  \\
 & & valeur absolue. \\
 \end{tabular}
\end{exemple*1}


\exercice
Range dans l'ordre croissant les nombres suivants : 
\begin{colenumerate}{2}
 \item $+12$ ; 0 ; $-7$ ; $-5$ ; $+5$ ;
 \item $-8$ ; $+10$ ; $-14$ ; $-21$ ; $+3$ ; $-1$ ;
 \item $-24$ ; $-2,4$ ; $2,4$ ; 0 ; $-4,2$ ; $-4$ ;
 \item $-2,4$ ; $+2,3$ ; $-2,42$ ; $+2,33$ ; $-3,23$.
 \end{colenumerate}
%\correction

\end{methode*1}
%%%%%%%%%%%%%%%%%%%%%%%%%%%%%%%%%%%%%%%%%%%%%%%%%%%%%%%%%%%%%%%%%
\newpage


\section{Repérage dans un plan}

\vspace{3em}

\begin{definition}
Dans un plan muni d'un repère, tout point est repéré par un couple de nombres relatifs appelé ses \MotDefinition{coordonnées}{} : la première est l'\textcolor{C2}{\textbf{abscisse}} (déplacement horizontal) et la seconde est l'\textcolor{C2}{\textbf{ordonnée}} (déplacement vertical).\\
Pour un point A, on écrira toujours \textbf{A(abscisse de A ; ordonnée de A)}.
\end{definition}

\vspace{3em}

\begin{methode*1}[Repérer un point dans un plan]

\begin{exemple*1}
Lis les coordonnées du point $A$ et du point $B$ puis place les points $C(5 ; -3)$ et $D(-3 ; 0)$ :

\begin{center} 
\begin{tikzpicture}[general,scale=0.5]
\draw[xstep=1,ystep=1,color=gray!80] (-6,-4) grid (6,4);
\axeX{-6}{6}{1}
\axeY{-4}{4}{1}

\draw[ultra thick,loosely dotted,color=C1](-4,2)--(0,2);
\node[right,color=C1,font=\bfseries] at (0,2){$+2$};
\draw[ultra thick,loosely dotted,color=G1](-4,2)--(-4,0);
\node[below,color=G1,font=\bfseries] at (-4,0){$-4$};

\node at (0,-3){$\bullet$};
\node[above left,font=\bfseries] at (-4,2){A};
\node at (-4,2){$\bullet$};
\node[right,font=\bfseries] at (0,-3){B};
\end{tikzpicture}
\end{center}

Pour lire les coordonnées du point $A$, on repère l'abscisse de $A$ sur l'axe horizontal (pointillés bleus) puis  son ordonnée sur l'axe vertical (pointillés violets). On conclut en donnant l'abscisse puis l'ordonnée : $A (-4 ; +2)$. \\[0.5em]
Le point $B$ appartient à l'axe des ordonnées donc son abscisse est 0. Ses coordonnées sont $(0 ; -3)$.

Pour placer le point $C$, on repère tous les points d'abscisse $+5$ puis on repère tous les points d'ordonnée $-3$. On place le point $C$ à l'intersection des deux lignes. \\[0.5em]
L'ordonnée du point $D$ est 0 donc $D$ appartient à l'axe des abscisses.
\end{exemple*1}

\exercice 

\begin{minipage}[c]{0.45\linewidth}
Sur la figure ci-contre, lis les coordonnées des points $K$, $L$, $M$, $N$, $P$ et $R$ :
 \end{minipage} \hfill%
 \begin{minipage}[c]{0.4\linewidth}
 \begin{center} 
\begin{tikzpicture}[general,scale=0.5]
\draw[xstep=1,ystep=1,color=gray!80] (-6,-4) grid (6,3);
\axeX{-6}{6}{1}
\axeY{-4}{3}{1}

\node at (-3,2){$\bullet$};
\node[left,font=\bfseries] at (-3,2){K};
\node at (-5,0){$\bullet$};
\node[below,font=\bfseries] at (-5,0){L};
\node at (2,-3){$\bullet$};
\node[left,font=\bfseries] at (2,-3){M};
\node at (0,1){$\bullet$};
\node[right,font=\bfseries] at (0,1){N};
\node at (4,-1){$\bullet$};
\node[left,font=\bfseries] at (4,-1){P};
\node at (-4,-3){$\bullet$};
\node[left,font=\bfseries] at (-4,-3){R};
\end{tikzpicture}
\end{center}
  \end{minipage} \\
%\correction

\exercice Trace sur ton cahier un repère d'origine $O$. L'unité de longueur est le centimètre sur les deux axes. Place les points suivants :
\begin{colenumerate}{4}
 \item $E(+2 ; +3)$ ;
 \item $F(-2 ; -3)$ ;
 \item $G(+2 ; -3)$ ;
 \item $H(-2 ; 3)$.
 \end{colenumerate}
%\correction

\end{methode*1}




