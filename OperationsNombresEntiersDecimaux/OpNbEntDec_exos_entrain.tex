 \definecolor{fondTI}{HTML}{869286}
%%%%%%%%%%%%%%%%%%%%%%%%%%%%%%%%%%%%%%%%%%%%%%%%%%%%%%%%%%%%%%%%%%%%%%%%%%%
\serie{Techniques opératoires}
\begin{exercice}
Calcule mentalement les additions :
\begin{enumerate} 
 \item $4,6 + 5,2$ \dotfill ; 
 
 \item $6,2 + 3,4$ \dotfill ; 
 
 \item $4,5 + 6,1$ \dotfill ; 
 
 \item $8,3 + 9,6$ \dotfill ; 
 
 \item $8 + 1,5$ \dotfill ; 
 
 \item $8,6 + 8,9$ \dotfill ; 
 
 \item $3,9 + 5,4$ \dotfill ; 
 
 \item $6,5 + 8,7$ \dotfill ; 
 
 \item $6,8 + 9,4$ \dotfill ; 
 
 \item \hspace{0.1em} $12,9 + 15,8$ \dotfill. 
 \end{enumerate}
\end{exercice}
\begin{exercice}
Calcule mentalement les soustractions :
\begin{enumerate} 
 \item $6,5 - 4,3$ \dotfill ; 
 
 \item $7,6 - 0,4$ \dotfill ; 
 
 \item $4,9 - 4,3$ \dotfill ; 
 
 \item $5,7 - 0,4$ \dotfill ; 
 
 \item $4,7 - 4,3$ \dotfill ; 
 
 \item $6,2 - 4,6$ \dotfill ; 
 
 \item $9 - 8,7$ \dotfill ; 
 
 \item $3,1 - 1,8$ \dotfill ; 
 
 \item $7,8 - 6,9$ \dotfill ; 
 
 \item \hspace{0.2em}$17,4 - 8,7$ \dotfill. 
 
 \end{enumerate}  
\end{exercice}
\begin{exercice}
Calcule les sommes en effectuant des regroupements astucieux :
\begin{enumerate} 
 \item $6,5 + 12,6 + 1,5$ ;
 \item $36,99 + 45,74 + 2,01 + 13,26$ ;
 \item $9,25 + 8,7 + 5,3 + 16,75$ ;
 \item $34,645 + 34,75 + 2,25 + 4,355$ ;
 \item $7,42 + 4,2 + 7,8 + 25,58$ ;
 \item $3,01 + 2,9 + 6,1 + 7,99 + 2,001$.
 \end{enumerate}
\end{exercice}
\begin{exercice}
Pose et effectue :
\begin{enumerate} 
 \item $853,26 + 4 038,3$ ;
 \item $52 + 8,63 + 142,8$ ;
 \item $49,3 + 7,432 + 12,7$ ;
 \item $948,25 - 73,2$ ;
 \item $9,8 - 0,073$ ;
 \item $83 - 43,51$.
 \end{enumerate} 
 \end{exercice}
\begin{exercice}
Calcule mentalement :
\begin{enumerate} 
 \item $435,7 \times 0,1$ \dotfill ; 
 
 \item $18,73 \times 0,01$ \dotfill ; 
 
 \item $439,345 \times 0,001$ \dotfill ; 
 
 \item $0,28 \times 0,1$ \dotfill ; 
 
 \item $39 \times 0,001$ \dotfill ; 
 
 \item $0,8 \times 0,01$ \dotfill ; 
 
 \item $354 \times 0,001$ \dotfill ; 
 
 \item $0,03 \times 0,001$ \dotfill. 
 
 \end{enumerate}
\end{exercice}
\begin{exercice}
Calcule mentalement :
\begin{enumerate} 
 \item $48 \div 0,1$ \dotfill ; 
 
 \item $12,97 \div 0,01$ \dotfill ; 
 \item $12,3 \div 0,001$ \dotfill ; 
 
 \item $0,45 \div 0,1$ \dotfill ; 
        
 \item $5,61 \div 0,0001$ \dotfill ; 
        
 \item $0,056 \div 0,1$ \dotfill ; 
 
 \item $354 \div 0,001$ \dotfill ; 
 
 \item $0,5 \div 0,001$ \dotfill. 
 \end{enumerate}
\end{exercice}
\begin{exercice}
Complète par le signe opératoire qui convient :
\begin{colenumerate}{2}
 \item $0,8 \ldots 100 = 80$ ;
 \item $0,38 \ldots 10 = 0,038$ ;
 \item $47 \ldots 100 = 0,47$ ;
 \item $380 \ldots 10 = 38$ ;
 \item $5 \ldots 0,1 = 0,5$ ;
 \item $60\,000 \ldots 10 = 6\,000$ ;
 \item $4\,100 \ldots 100 = 4\,000$ ;
 \item $5\,600 \ldots 100 = 56$ ;
 \item $8 \ldots 0,01 = 0,08$ ;
 \item \hspace{0.25em}$100 \ldots 1,2 = 120$.
 \end{colenumerate} 
\end{exercice}
\begin{exercice}
Calcule mentalement en détaillant ta démarche :
\begin{enumerate} 
 \item $0,1 \times 14 \times 1\,000$ \dotfill ; 
 
 \item $2,18 \times 0,001 \times 100$ \dotfill ; 
 \item $1,8 \times 0,01 \times 10$ \dotfill ; 
 \item $4 •\times 0,01 \times 100$ \dotfill. 
 \end{enumerate} 
\end{exercice}
\begin{exercice}
Sachant que $48 \times 152 = 7\,296$, détermine les résultats des calculs :
\begin{enumerate} 
 \item $48 \times 1,52$ \dotfill ; 
 
 \item $4,8 \times 15,2$ \dotfill ; 
 
 \item $0,48 \times 0,152$ \dotfill ; 
 
 \item $0,048 \times 1\,520$ \dotfill. 
 \end{enumerate} 
\end{exercice}
\begin{exercice}
Calcule en regroupant astucieusement :
\begin{enumerate} 
 \item $0,8 \times 2 \times 0,6 \times 50$ \dotfill ; 
 
 \item $0,25 \times 12,38 \times 4$ \dotfill ; 
 
 \item $8 \times 49 \times 1,25$ \dotfill ; 
 
 \item $2,5 \times 12,9 \times 0,04$ \dotfill ; 
 
 \item $0,15 \times 70 \times 0,02$ \dotfill ; 
 
 \item $75 \times 0,06 \times 0,4$ \dotfill. 
 
 \end{enumerate} 
\end{exercice}
\begin{exercice}
Place correctement la virgule dans le résultat de la multiplication (en ajoutant éventuellement un ou des zéros) :
\begin{enumerate} 
 \item $12,8 \times  5,3 = \textcolor{PartieGeometrie}{6\,784}$ ;
 \item $28,7 \times 1,04 = \textcolor{PartieGeometrie}{29\,848}$ ;
 \item $0,15 \times 6,3 = \textcolor{PartieGeometrie}{945}$ ;
 \item $0,008 \times 543,9 = \textcolor{PartieGeometrie}{43\,512}$ ;
 \item $0,235 \times 0,132 = \textcolor{PartieGeometrie}{3\,102}$.
 \end{enumerate}
\end{exercice}
\begin{exercice}
Pose et effectue les produits :
\begin{enumerate} 
 \item $2,08 \times 4,23$ \dotfill ; 
 
 \item $4,38 \times 5,7$ \dotfill ; 
 
 \item $6,93 \times 15,8$ \dotfill ; 
 
 \item $8,35 \times 0,18 $\dotfill.  
 \end{enumerate}
\end{exercice}
\begin{exercice} 
Calcule mentalement :
\begin{colenumerate}{2}
 \item $ 8,6 \div 2$ ;
 \item $ 24,8 \div 4$ ;
 \item $ 8,8 \div 8$ ;
 \item $ 7,7 \div 11$ ;
 \item $ 15,6 \div 3$ ;
 \item $ 63,6 \div 6$.
 \end{colenumerate}
\end{exercice}
\begin{exercice} 
Pose et effectue les divisions suivantes pour en trouver le quotient décimal exact :
\begin{enumerate} 
 \item $ 12,6 \div 6$ \dotfill ; 
 
 \item $ 28,48 \div 4$ \dotfill ; 
 \item $ 169,2 \div 3$ \dotfill ; 
 \item $ 0,162 \div 9$ \dotfill ; 
 \item $ 67,5 \div 4$ \dotfill ; 
 \item $ 9,765 \div 15$ \dotfill. 
 \end{enumerate}
\end{exercice}
\begin{exercice}[Valeurs approchées]
\vspace{-1em}
\begin{enumerate} 
 \item Pose et effectue les divisions suivantes jusqu'au millième :
 \begin{itemize}
  \item $12 \div 7$ \dotfill ; 
  
  \item $148,9 \div 12$ \dotfill ; 
  
  \item $13,53 \div 3$ \dotfill. 
  \end{itemize}
 \item Pose et effectue les divisions suivantes jusqu'au centième :
  \begin{itemize}
  \item $123,8 \div 7$ \dotfill ; 
  
  \item $235,19 \div 11$ \dotfill ; 
  
  \item $0,14 \div 3$ \dotfill. 
  \end{itemize}
 \end{enumerate}
\end{exercice}
\begin{exercice} 
Calcule la valeur exacte ou une valeur arrondie au centième des divisions suivantes :
\begin{enumerate} 
 \item $1 \div 2,74$ \dotfill ; 
 \item $5,87 \div 2,3$ \dotfill ; 
 \item $3,24 \div 1,7$ \dotfill ; 
 \item $45,6 \div 0,24$ \dotfill ; 
 \item $20,35 \div 8,5$ \dotfill ; 
 \item $0,53 \div 0,17$ \dotfill. 
 \end{enumerate}
\end{exercice}
\begin{exercice} 
Calcule la valeur exacte ou une valeur arrondie au centième des divisions suivantes :
\begin{enumerate} 
 \item $3,35 \div 0,42$ \dotfill ; 
 \item $41,5 \div 3,14$ \dotfill ; 
 \item $ 0,03 \div 2,1$ \dotfill ; 
 \item $0,35 \div 0,25$ \dotfill ; 
 \item $0,53 \div 0,8$ \dotfill ; 
 \item $21,7 \div 0,14$ \dotfill. 
 \end{enumerate}
\end{exercice}
%%%%%%%%%%%%%%%%%%%%%%%%%%%%%%%%%%%%%%%%%%%%%%%%%%%%%%%%%%%%%%%%%%%%%%%%%%%
%%%%%%%%%%%%%
\serie{Techniques opératoires}
\begin{exercice}
Calcule mentalement :
\begin{enumerate} 
 \item $4,357 \times 100$ \dotfill ; 
 
 \item $89,7 \times 1\,000$ \dotfill ; 
 
 \item $0,043 \times 10$ \dotfill ; 
 
 \item $0,28 \times 1\,000$ \dotfill ; 
 
 \item $39 \times 100$ \dotfill ; 
 
 \item $0,48 \times 10$ \dotfill ; 
        
 \item $354 \times 10$ \dotfill ; 
        
 \item $0,03 \times 10\,000$ \dotfill ; 
 
 \end{enumerate}
\end{exercice}
\begin{exercice}
Calcule mentalement :
\begin{enumerate} 
 \item $4\,338 \div 10$ \dotfill ; 
 
 \item $1\,297 \div 1\,000$ \dotfill ; 
        
 \item $12,3 \div 10$ \dotfill ; 
 
 \item $0,87 \div 100$ \dotfill ; 
        
 \item $3,8 \div 1\,000$ \dotfill ; 
 
 \item $0,04 \div 100$ \dotfill ; 
        
 \item $354 \div 10$ \dotfill ; 
 
 \item $12,5 \div 100$ \dotfill. 
 
 \end{enumerate}
\end{exercice}
\begin{exercice}
Complète par 10 ; 100 ; 1\,000 ; 10\,000 \ldots :
\begin{enumerate} 
 \item $8,79 \times \dotfill = 87,9$ ; 
 
 \item $4,35 \times \dotfill = 43\,500$ ; 
 
 \item $0,837 \times \dotfill = 8,37$ ; 
 
 \item $0,367 \times \dotfill = 3,67$ ; 
 
 \item $0,028 \times \dotfill = 0,28$ ; 
 
 \item $0,17 \div \dotfill = 0,017$ ; 
 
 \item $23 \div \dotfill = 0,23$ ; 
 
 \item $480 \div \dotfill = 4,8$ ; 
 
 \item $900 \div \dotfill = 0,09$ ; 
 
 \item \hspace{0.25em}$18\,000 \div \dotfill = 18$. 
 
 \end{enumerate}
\end{exercice}
\begin{exercice}
Place la virgule dans le nombre écrit en \textcolor{BleuOuv}{bleu} pour que l'égalité soit vraie :
\begin{enumerate} 
 \item $3,42 \times \textcolor{BleuOuv}{271} = 9,268\,2$ ;
 \item $\textcolor{BleuOuv}{432} \times 0,614 = 26,524\,8$ ;
 \item $0,48 \times \textcolor{BleuOuv}{62} = 29,76$ ;
 \item $2,6 \times \textcolor{BleuOuv}{485} = 126,1$ ;
 \item $\textcolor{BleuOuv}{45} \times 29,232 = 131,544$.
 \end{enumerate}
\end{exercice}
\begin{exercice}
Pose et effectue les produits :
\begin{enumerate} 
 \item $2,08 \times 4,23$ \dotfill ; 
 
 \item $4,38 \times 5,7$ \dotfill ; 
 
 \item $6,93 \times 15,8$ \dotfill ; 
 
 \item $8,35 \times 0,18 $\dotfill.  
 \end{enumerate}
\end{exercice}
