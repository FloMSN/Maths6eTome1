\begin{activite}[Segments, droites et demi-droites]

\section{Découvrir les outils de GeoGebra}
	\subsection{les principaux outils}
Commencer par lancer le logiciel GeoGebra. Faire afficher une page blanche à l'aide du menu "affichage", en déselectionnant les axes et/ou grille si besoin.

En parcourant les différents outils de construction disponibles dans la barre d'outils, reliez proprement cahque outil avec l'icône qui lui correspond.

\begin{center}
 \begin{tabularx}{\linewidth}{|r|lXrc}
  \cline{1-1}
  Nouveau point & \huge{\textbullet} & & \huge{\textbullet} & \includegraphics[width=1cm]{geopolygone} \\  \cline{1-1}
  Déplacer la feuille de travail & \huge{\textbullet} & & \huge{\textbullet} & \includegraphics[width=1cm]{geoangle} \\ \cline{1-1}
  Demi-droite passsant par deux points & \huge{\textbullet} & & \huge{\textbullet} & \includegraphics[width=1cm]{geobissectrice} \\ \cline{1-1}
  Droite perpendiculaire & \huge{\textbullet} & & \huge{\textbullet} & \includegraphics[width=1cm]{geosegment} \\ \cline{1-1}
  Angle & \huge{\textbullet} & & \huge{\textbullet} & \includegraphics[width=1cm]{geofleche} \\ \cline{1-1}
  Milieu ou centre & \huge{\textbullet} & & \huge{\textbullet} & \includegraphics[width=1cm]{geopoint} \\ \cline{1-1}
  Droite passant par deux points & \huge{\textbullet} & & \huge{\textbullet} & \includegraphics[width=1cm]{geomediatrice} \\ \cline{1-1}
  Segment entre deux points & \huge{\textbullet} & & \huge{\textbullet} & \includegraphics[width=1cm]{geodemidroite} \\ \cline{1-1}
  Déplacer & \huge{\textbullet} & & \huge{\textbullet} & \includegraphics[width=1cm]{geodroite} \\ \cline{1-1}
  Bissectrice & \huge{\textbullet} & & \huge{\textbullet} & \includegraphics[width=1cm]{geodeplacement} \\ \cline{1-1}
  Médiatrice & \huge{\textbullet} & & \huge{\textbullet} & \includegraphics[width=1cm]{geoperpendiculaire} \\ \cline{1-1}
  Polygone & \huge{\textbullet} & & \huge{\textbullet} & \includegraphics[width=1cm]{geomilieu} \\ \cline{1-1}
  \end{tabularx}
\end{center}

	\subsection{Construire sa première figure}
En utilisant les commandes de l'exercice précédent, réalise la figure correspondant au programme de construction suivant:
\begin{enumerate}
\item Créer deux points A et B et tracer la droite (AB).
\item Déplacer le point A. La droite (AB) doit se déplacer en suivant le point.
\item Placer un point C tel que $C \notin(AB)$. Tracer la demi-droite [AC) et le segment [BC].
\item Afficher la longeur BC.
\item Placer le milieu I du segment [BC] et afficher le longueur BI.
\item Tracer le cercle de centre A passant par B.
\item Tracer le cercle de centre C, de rayon 4.
\item Fais vérifier ton travail par le professeur.
\end{enumerate}

	\subsection{Reproduction d'une figure}
\begin{enumerate}
\item Réaliser la figure ci-contre (B est à l'intersection des deux cercles)
\item Déplacer le point B. S'il reste sur les deux cercles alors la figures est juste.
\item Fais vérifier ton travail par le professeur.
\end{enumerate}

\end{activite}

%%%%%%%%%%%%%%%%%%%%%%%%%%%%%%%%%%%%%%%%%%%%%%%%%%%%%%%%%%%%%%%%%%%%%%%%%%%%%%%%%%%%%%%%%%%%%%%%%%


\begin{activite}[À la découverte d'un nouveau code]

  \begin{enumerate}
   \item Lire la consigne de la case \circled{1} et observer la figure correspondant à cette consigne.
Faire de même pour la case \circled{2}.

Quand le code est compris, tracer la figure de la case \circled{3} et écrire la consigne de la case \circled{4}.


  \vspace{1em}
  
  
  
  \begin{tabular}{|c|c|c|c|c|}
  \cline{1-2}\cline{4-5}
    \circled{1} 		& \circled{2} 		& 	& \circled{3} 		& \circled{4}	\\ 
     Tracer $(AB)$ 	&  Tracer $[AC)$ 	& 	& Tracer $(AB)$	& 	 		\\ 
     Tracer $[AC]$ 	& Tracer $[BC]$ 	& 	& Tracer$[BC]$		& 			\\
     				&				&	& Tracer$[AC)$		&			\\ \cline{1-2}\cline{4-5}
   \includegraphics[width=.2\linewidth]{tracerAB-AC} & 
   \includegraphics[width=.2\linewidth]{tracerAC-BC} & & 
   \includegraphics[width=.2\linewidth]{tracerAB-BC-AC}& 
   \includegraphics[width=.2\linewidth]{tracer} 					\\ \cline{1-2}\cline{4-5}
  \end{tabular}\\[1em]

  
   \item Lire la consigne de la case \circled{5} et observer la figure correspondant à cette consigne. Tracer ensuite la figure de la case \circled{6} et écrire la consigne de la case \circled{7}.
   
   \vspace{1em}
  
    \begin{tabular}{|l|l|l|}
   \hline
    \hfill \circled{5} \hfill			&	\hfill \circled{6} \hfill				&	\hfill \circled{7} \hfill 	\\
    - Tracer la droite passant par 	&	- Tracer le segment 				&					\\
    $E$ et $F$ ;					&	d'extrémités $R$ et $S$ ;			&					\\
    - Tracer le segment 			&	- Tracer la droite passant par 		&					\\
    d'extrémités $E$ et $G$ ;		&	$R$ et $T$ ;					&					\\
    - Tracer la demi-droite 			&	- Tracer la demi-droite 			&					\\
    d'origine $G$ et passant par $F$.	&	d'origine $S$ et passant par $T$.	&					\\ \hline
    \includegraphics[width=.24\linewidth]{tracerEFG} 			&  
    \includegraphics[width=.24\linewidth]{tracerRST}			&
    \includegraphics[width=.24\linewidth]{tracerLCI}			\\ \hline
    \end{tabular}\\[1em]

    
  \newpage
  
   \item Compléter le tableau suivant :
   
   \vspace{1em}
   
   \renewcommand*\tabularxcolumn[1]{>{\centering\arraybackslash}m{#1}}
   \begin{ttableau}{\linewidth}{3}
    \hline
    \multicolumn{1}{|c|}{\textbf{Phrase}}	&	\multicolumn{1}{c}{\textbf{Phrase codée}}	&	\multicolumn{1}{|c|}{\textbf{Dessin}}			 	\\  \hline
    								&	Tracer $[UV]$							&	\includegraphics[width=2.6cm]{phraseUV}		\\  \hline
    								&										&	\includegraphics[width=3.2cm]{phraseAM}		\\  \hline
   Tracer la droite passant par $S$ et $T$	&										&	\includegraphics[width=2.6cm]{phraseST}		\\  \hline
   								&										&	\includegraphics[width=4.0cm]{phraseAM_2}	\\  \hline
   Tracer le segment d'extrémités $M$ et $N$	&									&	\includegraphics[width=2.6cm]{phraseMN} 	\\  \hline
   								&	Tracer $[KJ)$							&	\includegraphics[width=2.6cm]{phraseKJ}		\\  \hline
								&										&	\includegraphics[width=2.6cm]{phraseAM_3}	\\  \hline
   Tracer la demi-droite d'origine $O$ et 	&										&	\includegraphics[width=2.6cm]{phraseOU}		\\
   passant par $U$					&										&										\\  \hline
   								&	Tracer $(BC)$							&	\includegraphics[width=2.6cm]{phraseBC}		\\  \hline
  \end{ttableau}
  
   \end{enumerate} 

\end{activite}
%%%%%%%%%%%%%%%%%%%%%%%%%%%%%%%%%%%%%%%%%%%%%
