%manuel 6e, chapitre G3
\begin{tikzpicture}[scale=0.5,every node/.style={scale=1},rotate=30]

\draw (0,2) node [left]{M};
\draw (0,2) node {$\times$};
\draw (-2,0)--(3,0) node [near end,below] {$(d)$};
\draw [thick](-0.02,4.2)--(-0.02,2.45);

%%%%%%%%%%%%%%%%%%%%%%%%
%%%%%%%%%%%%%%%%%%%%%%%%
%Définition des paramètres de l'équerre
%et de son positionnement
%%%%%%%%%%%%%%%%%%%%%%%%
%%%%%%%%%%%%%%%%%%%%%%%%

\def \xorigine {0}; %abscisse de l'origine de l'équerre posée avec un xshift
\def \yorigine {0}; %ordonnée de l'origine de l'équerre posée avec un yshift
\def \rotation {0}; %angle de rotation de l'équerre
\def \longueur {4}; %longueur de l'équerre
\def \largeur {2}; %largeur de l'équerre
\def \epaisseur {\longueur * 0.1}; %épaisseur de la partie «colorée» de l'équerre

%%%%%%%%%%%%%%%%%%%%%%%%
%%%%%%%%%%%%%%%%%%%%%%%%
%Tracé de l'équerre
%%%%%%%%%%%%%%%%%%%%%%%%
%%%%%%%%%%%%%%%%%%%%%%%%

\begin{scope}[scale=1.1,xshift=\xorigine cm,yshift=\yorigine cm,rotate=\rotation]

%contour extérieur de l'équerre
\coordinate (A) at (0,0) ; %«origine» de l'équerre
\coordinate (B) at (\largeur,0) ;
\coordinate (C) at (0,\longueur) ;
\draw [gray](A)--(B)--(C)--cycle;


%contour intérieur de l'équerre
\coordinate (D) at (\epaisseur,\epaisseur) ;
\coordinate (E) at ($\largeur*(1,0)-{\largeur * \epaisseur / \longueur}*(1,0)-\epaisseur*(1,0)+\epaisseur*(0,1)$);
\coordinate (F) at ($\epaisseur*(1,0)+\longueur*(0,1)-{2*\longueur * \epaisseur / \largeur}*(0,1)$);
\draw [gray](D)--(E)--(F)--cycle;

%partie colorée de l'équerre
\fill [color=blue!50!gray,opacity=.4,even odd rule] (A)--(B)--(C)--cycle (D)--(E)--(F)--cycle;%l'option even odd rule permet de faire le remplissage entre les 2 zones définies
\end{scope}

%%%%%%%%%%%%%%%%%%%%%%%%
%%%%%%%%%%%%%%%%%%%%%%%%
%Fin de l'équerre
%%%%%%%%%%%%%%%%%%%%%%%%
%%%%%%%%%%%%%%%%%%%%%%%%


%%%%%%%%%%%%%%%%%%%%%%%%
%%%%%%%%%%%%%%%%%%%%%%%%
%Début crayon
%%%%%%%%%%%%%%%%%%%%%%%%
%%%%%%%%%%%%%%%%%%%%%%%%

\begin{scope}[scale=0.35,xshift=0cm,yshift=7cm,rotate=-70] %le crayon, xshift et yshift pour les coordonnées de la pointe, rotate pour l'orientation du crayon
\def \couleur {black}
\coordinate (O) at (0,0);
\fill[\couleur!40] (-0.2,4.8) -- (0.2,4.8) -- (0.2,0.8) --(0.1,0.65) -- (0,0.8) -- (-0.1,0.66) -- (-0.2,0.8) -- cycle; %corps du crayon
\draw[color=white] (0,4.8) -- (0,0.8); %trait intérieur du crayon
\fill[\couleur!90] (-0.2,4.3) -- (0,4.27) -- (0.2,4.3) -- (0.2,4.8) arc(30:150:0.23cm); %partie haute du crayon
\fill[brown!40] (-0.2,0.8) -- (O)node[coordinate,pos=0.75](a){} -- (0.2,0.8)node[coordinate,pos=0.25](b){} -- (0.1,0.65) -- (0,0.8) -- (-0.1,0.66) -- cycle; %pointe du crayon (partie taillée)
\fill[\couleur!90] (a) -- (O) -- (b) -- cycle; %mine du crayon
\end{scope}

%%%%%%%%%%%%%%%%%%%%%%%%
%%%%%%%%%%%%%%%%%%%%%%%%
%Fin crayon
%%%%%%%%%%%%%%%%%%%%%%%%
%%%%%%%%%%%%%%%%%%%%%%%%


\end{tikzpicture} 
