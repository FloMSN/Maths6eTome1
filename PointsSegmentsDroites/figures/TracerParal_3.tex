%manuel 6e, chapitre G3
\begin{tikzpicture}[scale=.5,every node/.style={scale=1},rotate=30]

\draw (0,2) node [above left]{M};
\draw (0,2) node {$\times$};
\draw (-2.2,0)--(2.6,0) node [near end,below] {$(d)$};
\draw [very thick,red,dashed,->,>=stealth] (-0.5,0)--(-0.5,2);


%%%%%%%%%%%%%%%%%%%%%%%%
%%%%%%%%%%%%%%%%%%%%%%%%
%Définition des paramètres de l'équerre
%et de son positionnement
%%%%%%%%%%%%%%%%%%%%%%%%
%%%%%%%%%%%%%%%%%%%%%%%%

\def \xorigine {-1.65}; %abscisse de l'origine de l'équerre posée avec un xshift
\def \yorigine {0}; %ordonnée de l'origine de l'équerre posée avec un yshift
\def \rotation {-90}; %angle de rotation de l'équerre
\def \longueur {4}; %longueur de l'équerre
\def \largeur {2}; %largeur de l'équerre
\def \epaisseur {\longueur * 0.1}; %épaisseur de la partie «colorée» de l'équerre

%%%%%%%%%%%%%%%%%%%%%%%%
%%%%%%%%%%%%%%%%%%%%%%%%
%Tracé de l'équerre
%%%%%%%%%%%%%%%%%%%%%%%%
%%%%%%%%%%%%%%%%%%%%%%%%

\begin{scope}[scale=.6,xshift=\xorigine cm,yshift=\yorigine cm,rotate=\rotation]

%contour extérieur de l'équerre
\coordinate (A) at (0,0) ; %«origine» de l'équerre
\coordinate (B) at (\largeur,0) ;
\coordinate (C) at (0,\longueur) ;
\draw [gray, opacity=0.2](A)--(B)--(C)--cycle;


%contour intérieur de l'équerre
\coordinate (D) at (\epaisseur,\epaisseur) ;
\coordinate (E) at ($\largeur*(1,0)-{\largeur * \epaisseur / \longueur}*(1,0)-\epaisseur*(1,0)+\epaisseur*(0,1)$);
\coordinate (F) at ($\epaisseur*(1,0)+\longueur*(0,1)-{2*\longueur * \epaisseur / \largeur}*(0,1)$);
\draw [gray, opacity=0.2](D)--(E)--(F)--cycle;

%partie colorée de l'équerre
\fill [color=blue!50!gray,opacity=.1,even odd rule] (A)--(B)--(C)--cycle (D)--(E)--(F)--cycle;%l'option even odd rule permet de faire le remplissage entre les 2 zones définies

\end{scope}
%%%%%%%%%%%%%%%%%%%%%%%%
%%%%%%%%%%%%%%%%%%%%%%%%
%Fin de l'équerre
%%%%%%%%%%%%%%%%%%%%%%%%
%%%%%%%%%%%%%%%%%%%%%%%%


%%%%%%%%%%%%%%%%%%%%%%%%
%%%%%%%%%%%%%%%%%%%%%%%%
%Définition des paramètres de l'équerre
%et de son positionnement
%%%%%%%%%%%%%%%%%%%%%%%%
%%%%%%%%%%%%%%%%%%%%%%%%

\def \yorigine2 {3.3}; %ordonnée de l'origine de l'équerre posée avec un yshift

%%%%%%%%%%%%%%%%%%%%%%%%
%%%%%%%%%%%%%%%%%%%%%%%%
%Tracé de l'équerre
%%%%%%%%%%%%%%%%%%%%%%%%
%%%%%%%%%%%%%%%%%%%%%%%%

\begin{scope}[scale=.6,xshift=\xorigine cm,yshift=\yorigine2 cm,rotate=\rotation]

%contour extérieur de l'équerre
\coordinate (A) at (0,0) ; %«origine» de l'équerre
\coordinate (B) at (\largeur,0) ;
\coordinate (C) at (0,\longueur) ;
\draw [gray](A)--(B)--(C)--cycle;


%contour intérieur de l'équerre
\coordinate (D) at (\epaisseur,\epaisseur) ;
\coordinate (E) at ($\largeur*(1,0)-{\largeur * \epaisseur / \longueur}*(1,0)-\epaisseur*(1,0)+\epaisseur*(0,1)$);
\coordinate (F) at ($\epaisseur*(1,0)+\longueur*(0,1)-{2*\longueur * \epaisseur / \largeur}*(0,1)$);
\draw [gray](D)--(E)--(F)--cycle;

%partie colorée de l'équerre
\fill [color=blue!50!gray,opacity=.4,even odd rule] (A)--(B)--(C)--cycle (D)--(E)--(F)--cycle;%l'option even odd rule permet de faire le remplissage entre les 2 zones définies

\end{scope}
%%%%%%%%%%%%%%%%%%%%%%%%
%%%%%%%%%%%%%%%%%%%%%%%%
%Fin de l'équerre
%%%%%%%%%%%%%%%%%%%%%%%%
%%%%%%%%%%%%%%%%%%%%%%%%



%%%%%%%%%%%%%%%%%%%%%%%%
%%%%%%%%%%%%%%%%%%%%%%%%
%Début règle !! Si la figure est tournée, il faut
%rajouter l'angle de rotation dans le node des graduation
%pour que les nombres soient écrits correctement
%%%%%%%%%%%%%%%%%%%%%%%%
%%%%%%%%%%%%%%%%%%%%%%%% 

    %Graduaton max. de la règle
    \def \Taille {5}
    %Définition de l 'angle de rotation de la règle
    \def \Rotation {-90}
    %Définition du décalage de la règle
    \def \DecalX {-1.5}
    \def \DecalY {2.8}
    %Couleur des élèments de la règle (sauf le remplissage)
    \def \RegleColor {blue!60}

\begin{scope}[shift={(\DecalX,\DecalY)},rotate=\Rotation,scale=1]
    % contours de la règle
    \draw[color=\RegleColor, fill =blue!5, opacity=0.5] (-0.2,0.5) rectangle (\Taille+0.2,-0.5);	%Dont couleur de remplissage
    % graduation 1 mm
    \foreach \a in {0,0.1,...,\Taille}{\draw[color=\RegleColor] (\a,0.5)--(\a,0.42);}
    % graduation 5 mm
    \foreach \a in {0,0.5,...,\Taille}{\draw[color=\RegleColor] (\a,0.42)--(\a,0.35);}
    % graduation et repères 10 mm
    \foreach \a in {0,1,...,\Taille}{\draw[color=\RegleColor] (\a,0.35)--(\a,0.25)
    node[font=\tiny, rotate=\Rotation+30] (\a) at (\a,0.1){\a};}%ici rajouter l'angle de rotation de la figure complète pour que les nombres soient écrits correctement
\end{scope}

%%%%%%%%%%%%%%%%%%%%%%%%
%%%%%%%%%%%%%%%%%%%%%%%%
%Fin de la règle
%%%%%%%%%%%%%%%%%%%%%%%%
%%%%%%%%%%%%%%%%%%%%%%%%


\end{tikzpicture} 
