
% Attention, il faut déclarer la librairie calc :\usetikzlibrary{calc}  pour les calculs de coordonnées des points

\begin{tikzpicture}[scale=0.5,every node/.style={scale=1.3}]
  
%%%%%%%%%%%%%%%%%%%%%%%%
%%%%%%%%%%%%%%%%%%%%%%%%
%Définition des paramètres de l'équerre
%et de son positionnement
%%%%%%%%%%%%%%%%%%%%%%%%
%%%%%%%%%%%%%%%%%%%%%%%%

\def \xorigine {0}; %abscisse de l'origine de l'équerre posée avec un xshift
\def \yorigine {0}; %ordonnée de l'origine de l'équerre posée avec un yshift
\def \rotation {-20}; %angle de rotation de l'équerre
\def \longueur {4}; %longueur de l'équerre
\def \largeur {2}; %largeur de l'équerre
\def \epaisseur {\longueur * 0.1}; %épaisseur de la partie «colorée» de l'équerre

%%%%%%%%%%%%%%%%%%%%%%%%
%%%%%%%%%%%%%%%%%%%%%%%%
%Tracé de l'équerre
%%%%%%%%%%%%%%%%%%%%%%%%
%%%%%%%%%%%%%%%%%%%%%%%%
\begin{scope}[scale=0.8,xshift=\xorigine cm,yshift=\yorigine cm,rotate=\rotation]

%contour extérieur de l'équerre
\coordinate (A) at (0,0) ; %«origine» de l'équerre
\coordinate (B) at (\largeur,0) ;
\coordinate (C) at (0,\longueur) ;
\draw [gray](A)--(B)--(C)--cycle;


%contour intérieur de l'équerre
\coordinate (D) at (\epaisseur,\epaisseur) ;
\coordinate (E) at ($\largeur*(1,0)-{\largeur * \epaisseur / \longueur}*(1,0)-\epaisseur*(1,0)+\epaisseur*(0,1)$);
\coordinate (F) at ($\epaisseur*(1,0)+\longueur*(0,1)-{2*\longueur * \epaisseur / \largeur}*(0,1)$);
\draw [gray](D)--(E)--(F)--cycle;

%partie colorée de l'équerre
\fill [color=blue!50!gray,opacity=0.5,even odd rule] (A)--(B)--(C)--cycle (D)--(E)--(F)--cycle;%l'option even odd rule permet de faire le remplissage entre les 2 zones définies

\end{scope}

\begin{scope}[scale=0.5,xshift=.32cm,yshift=3cm,rotate=30] %le crayon
\fill[gray!50] (0,4) -- (0.4,4) -- (0.4,0) --(0.3,-0.15) -- (0.2,0) -- (0.1,-0.14) -- (0,0) -- cycle;
\draw[color=white] (0.2,4) -- (0.2,0);
\fill[black] (0,3.5) -- (0.2,3.47) -- (0.4,3.5) -- (0.4,4) arc(30:150:0.23cm);
\fill[brown!40] (0,0) -- (0.2,-0.8)node[coordinate,pos=0.75](a){} --(0.4,0) node[coordinate,pos=0.25](b){} -- (0.3,-0.15) -- (0.2,0) -- (0.1,-0.14) -- cycle;
\fill[black] (a) -- (0.2,-0.8) -- (b) -- cycle;
\end{scope}

\begin{scope}[rotate=-20] %la figure
\draw (-2,0)--(0,0) node [midway,red] {$\times$};
\draw (0,0)--(2,0) node [midway,red] {$\times$};
\draw (-2,0)--+(90:0.1)--+(-90:0.1);
\draw (2,0)--+(90:0.1)--+(-90:0.1);
\draw (0,0)--+(90:0.1)--+(-90:0.1);
\node [above right] at (-2,0) {O};
\node [above] at (2,0) {S};

\draw (0,0)--($(0,0)!1.3cm!90:(2,0)$); %trace un segment de 1.3cm à 90° du segment allant de (0,0) à (2,0)

\end{scope}



\end{tikzpicture} 
