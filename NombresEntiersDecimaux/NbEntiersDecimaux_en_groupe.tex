
\begin{TP}[]
Voici un extrait de « La Disme », écrit par Simon Stevin en 1585 : \\[0.3em]

« Les 27 \circled{0} 8 \circled{1} 4 \circled{2} 7 \circled{3} donnés, font ensemble 27 $\dfrac{8}{10}$, $\dfrac{4}{100}$, $\dfrac{7}{1\,000}$, ensemble 27 $\dfrac{847}{1\,000}$, et par même raison les 37 \circled{0} 6 \circled{1} 7 \circled{2} 5 \circled{3} valent 37 $\dfrac{675}{1\,000}$. Le nombre de multitude des signes, excepté \circled{0}, n'excède jamais le 9. Par exemple nous n'écrivons pas 7 \circled{1} 12 \circled{2}, mais en leur lieu 8 \circled{1} 2 \circled{2}. »

\partie{Simon Stevin}

Par groupe, en vous documentant, répondez aux questions suivantes.
\begin{enumerate}
 \item Où Simon Stevin a-t-il vécu ?
 \item Quels sont les domaines dans lesquels Simon Stevin a travaillé ? Faites la synthèse des réponses de chaque groupe.
 \end{enumerate}
 
\partie{La Disme} % Pourquoi cette ligne est si rentrée ?

\begin{enumerate}
\item Cherchez comment on écrit de nos jours le nombre 38 \circled{0} 6 \circled{1} 5 \circled{2} 7 \circled{3}.

Comparez avec les réponses des autres groupes.

\item Écrivez, à la manière décrite par Simon Stevin, les nombres $124 + \dfrac{7}{10} + \dfrac{5}{100}$ et 34,802.

Comparez avec les réponses des autres groupes.

 \item Choisissez trois nombres décimaux différents et écrivez-les à la manière décrite par Simon Stevin.
 \item échangez ensuite avec un autre groupe ces nombres écrits à la manière de Simon Stevin. Cherchez alors comment on écrit de nos jours les nombres que vous avez reçus.
 \item Faites une recherche pour trouver les différentes notations utilisées depuis 1585 pour l'écriture des nombres décimaux.
 \end{enumerate}
\end{TP}

%%%%%%%%%%%%%%%%%%%%%%%%%%%%%%%%%%%%%%%%%%%%%%%%%%%%%%%%%%%%%%%%%%%%%%%%%%%

\begin{TP}[Compétitions dans la classe]
Préparatifs : fabriquez une étiquette de carton pour chaque élève de la classe, comportant son nom et son prénom. Mélangez ces étiquettes.

Voici un exemple de liste de calculs à effectuer :
\begin{enumerate}
 \item $853,12 + 19,7$ ;
 \item $538,21 - 42,16$ ;
 \item $65,24 \cdot 7,38$ ;
 \item $68,37 : 3$.
 \end{enumerate}

\partie{Entraînement en individuel (appelé 1 contre 10)}
Pour chaque manche, un élève $A$ est tiré au sort à l'aide des étiquettes et passe au tableau où un seul calcul écrit est à effectuer. \\[0.5em]
L'élève $A$ l'effectue en public pendant que tous les autres cherchent chacun sur une feuille. \\[0.5em]
Dès qu'un élève a trouvé la réponse et a écrit le calcul, il lève la main. Le professeur surveille le tableau et circule dans la classe pour vérifier le travail de chaque élève. \\[0.5em]
Il compte à haute voix de 1 à 10 en ajoutant 1 chaque fois qu'un travail est considéré comme correct. \\[0.5em]
Arrivé à 10, si l'élève $A$ n'a pas trouvé, la classe a gagné la manche. Par contre, si l'élève $A$ trouve avant la fin du décompte à 10, c'est lui qui a gagné.

\partie{Par équipes (appelé 2 contre 5)}
On constitue des binômes équilibrés d'élèves.\\[0.5em]
Lors du tirage au sort, l'élève $A$ désigné passe au tableau accompagné de son coéquipier mais seul l'élève $A$ peut écrire. \\[0.5em]
On démarre la compétition comme dans le « 1 contre 10 » mais le professeur ne compte que jusqu'à 5. 
\end{TP}

\newpage
