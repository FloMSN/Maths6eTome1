\begin{exercice}[Nombres croisés]
Recopie et complète la grille à l'aide des nombres que tu trouveras grâce aux définitions :

\begin{center}
\begin{tabularx}{.5\linewidth}{r|c|c|c|c|c|}
\multicolumn{1}{c}{}& \multicolumn{1}{c}{\textbf{A}} & \multicolumn{1}{c}{\textbf{B}} & \multicolumn{1}{c}{\textbf{C}} & \multicolumn{1}{c}{\textbf{D}} & \multicolumn{1}{c}{\textbf{E}} \\ \cline{2-6}
\textbf{I} & & & & \cellcolor{black} & \\ \cline{2-6} 
\textbf{II} & & & & & \\ \cline{2-6} 
\textbf{III} & & \cellcolor{black} & & & \\ \cline{2-6} 
\textbf{IV} & & & & & \cellcolor{black} \\ \cline{2-6} 
\textbf{V} & \cellcolor{black} & & & & \\ \cline{2-6} 
\end{tabularx}
\end{center}

\vspace{0.75em}

\textbf{Horizontalement}

\textbf{I} : La partie entière de 328,54. Le chiffre des centièmes de 634,152.

\textbf{II} : Son chiffre des dizaines est le triple de celui des unités.

\textbf{III} : Le chiffre des dixièmes de 34. Arrondi à l'unité de 178,356.

\textbf{IV} : Entier compris entre 8\,000 et 9\,000.

\textbf{V} : Quarante-deux centaines.

\vspace{0.75em}

\textbf{Verticalement}

\textbf{A} : $(3 \times 1 000) + (5 \times 100) + (8 \times 1)$.

\textbf{B} : Le nombre de dixièmes dans 2,6. La partie entière de 2\,498 centièmes.

\textbf{C} : Quatre-vingt-six milliers et cent deux unités.

\textbf{D} : En additionnant tous les chiffres de ce nombre, on trouve 20.

\textbf{E} : Arrondi à l'unité de 536,57. Entier qui précède 1.

\end{exercice}


\begin{exercice}
Voici les résultats (en secondes), pour les hommes, du 100 m aux JO de Pékin en 2008 : \vspace{0.75em}

Martina : 9,93 ; Frater : 9,97 ; Burns : 10,01 ; Patton : 10,03 ; Bolt : 9,69 ; Powell : 9,95 ; Thompson : 9,89 ; Dix : 9,91.\vspace{0.75em}

Classe les coureurs dans l'ordre décroissant de leur résultat.
\end{exercice}


\begin{exercice}[À ordonner]
Range les nombres suivants dans l'ordre croissant : \vspace{0.75em}

25 unités et deux dixièmes ; 2\,504 centièmes; $25 + 2$ centièmes ; deux mille cinquante‑deux centièmes ; 20,54 ; 254 dixièmes.
\end{exercice}


\begin{exercice}[À placer]
En choisissant judicieusement la longueur d'une graduation, place précisément sur une demi‑droite graduée les points $A$, $B$, $C$, $D$ et $E$ d'abscisses respectives : \\[0.75em]
12,02 ; mille deux cent treize centièmes ; $12 + 7$ centièmes ; 1\,198 centièmes ; cent vingt-et-un dixièmes.
\end{exercice}


\begin{exercice}[Comparaison]
\begin{enumerate}
 \item Quel est le plus grand nombre décimal ayant un chiffre après la virgule et inférieur à 83 ?
 \item Quel est le plus petit nombre décimal avec trois chiffres après la virgule et supérieur à 214,3 ?
 \item Quel est le plus grand nombre décimal avec deux chiffres après la virgule, ayant tous ses chiffres différents et qui est inférieur à 97,8 ?
 \item Quel est le plus petit nombre décimal avec trois chiffres après la virgule, ayant tous ses chiffres différents et qui est supérieur à 2\,341 ?
 \end{enumerate}
\end{exercice}


\begin{exercice} 
Voici les masses de lipides et glucides (en g) contenues dans 50 g de différents biscuits :

\begin{center}
\begin{tabularx}{\linewidth}{|c|*{6}{>{\centering \arraybackslash}X|}}
\hline \rowcolor{U1} Biscuit & A & B & C & D & E \\
\hline \cellcolor{U1} Lipides & 9,527 & 9,514 & 9,53 & 9,521 & 9,6 \\
\hline \cellcolor{U1} Glucides & 32,43 & 33 & 33,6 & 33,15 & 33,50 \\
\hline
\end{tabularx} \\
\end{center}

\begin{enumerate}
 \item Classe ces biscuits selon l'ordre croissant de leur quantité de lipides ;
 \item Classe ces biscuits selon l'ordre décroissant de leur quantité de glucides.
 \end{enumerate}
\end{exercice}


\begin{exercice}[Énigme]
Trouve le nombre décimal à six chiffres tel que :
\begin{itemize}
 \item son chiffre des unités est 2 ;
 \item l'un de ses chiffres est 6 et sa valeur dans l'écriture décimale est cent fois plus petite que celle du chiffre 2 ;
 \item son chiffre des dizaines est le double de celui des unités et son chiffre des dixièmes est le quart de celui des dizaines ;
 \item ce nombre est compris entre 8\,975,06 et 9\,824,95 ;
 \item la somme de tous ses chiffres est égale à 27.
 \end{itemize}
\end{exercice}