\begin{enigme}[La constante de Champernowne]

Ce nombre, inventé par le mathématicien anglais David Gawen Champernowne en 1933, commence par

0,123456789101112131415 \ldots . \\[-1em]
\begin{enumerate}
 \item Quelle est la particularité de cette constante ? Donne les dix décimales suivantes.
 \item Quelle est l'arrondi, au cent-milliardième près, de cette constante ?
 \end{enumerate}
 \end{enigme}
        
\vspace*{2em}
        
\begin{enigme}[Défis]
Combien de fois faudrait-il utiliser le chiffre 1 si l'on voulait écrire tous les nombres entiers de 1 à 999 ? Et le chiffre 9 ?

Donne le nombre de mots utilisés pour écrire tous les entiers plus petits que 100.
 \end{enigme} 
 
 \vspace*{2em}

\begin{enigme}[Calculatrices infernales 1 (d'après Apmep)]
Sur la calculatrice d'Aïsha, la touche pour afficher la virgule ne fonctionne plus et la touche « $=$ » ne peut fonctionner qu'une seule fois par ligne de calcul.

Comment peut‑elle trouver le résultat de $(17,32 \cdot 45,3) + 15,437$ ?
 \end{enigme} 
 
 \vspace*{2em}

\begin{enigme}[Calculatrices infernales 2 (d'après Apmep)]
Bruce vient de faire tomber sa calculatrice. Elle ne comporte plus que les chiffres, la virgule et les quatre opérations, mais quand on appuie sur « $+$ » elle ajoute 1, quand on appuie sur « $-$ » elle retranche 1, quand on appuie sur la touche « $\times$ » elle multiplie par 10 et quand on appuie sur la touche « $\div$ » elle divise par 10. \\[-1em]
\begin{enumerate}
 \item Romain emprunte la calculatrice de Bruce. Il tape 27,2 puis appuie ensuite sur les touches « $\times$ », « $\times$ », « $+$ », « $+$ », « $-$ », « $\div$ », « $\div$ », « $\div$ », « $+$ », « $\times$ ». Quel résultat Romain trouve-t-il ?
 \item Comment peut‑il passer en sept opérations :    
 \begin{colitemize}{3}
  \item de 3,14 à 300 ?
  \item de 3,14 à 297 ?
  \item de 297 à 0,2 ?
  \end{colitemize}
 \item Tu viens de passer de 3,14 à 0,2 en quatorze opérations. Trouve un chemin qui permette de faire cela avec le minimum d'opérations. Compare avec tes camarades ; \\[0.5em]
Trouve un chemin qui permette de passer de 5 à 4,99 en un minimum d'opérations puis compare avec tes camarades.
 \end{enumerate}
\end{enigme} 
